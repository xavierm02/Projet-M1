\documentclass[11pt]{beamer}
\usetheme{Singapore}
\usepackage[utf8]{inputenc}
\usepackage[francais]{babel}
\usepackage[T1]{fontenc}
\usepackage{amsmath}
\usepackage{amsfonts}
\usepackage{amssymb}
\usepackage{tikz}
\usepackage{float}
\author{François Godi \and Xavier Montillet \and Chen Qian}
\title{Opacit\'e des automates}
\setbeamercovered{transparent} 
\setbeamertemplate{navigation symbols}{} 
%\logo{} 
%\institute{} 
%\date{} 
%\subject{} 
\bibliographystyle{plain}
\setbeamertemplate{bibliography item}{[\theenumiv]}

\usepackage{tcolorbox}
\tcbuselibrary{skins}

\colorlet{xlightblue}{blue!5}

\newtcolorbox{beamerlikethm}[1]{
  title=#1,
  beamer,
  colback=xlightblue,
  colframe=blue!30,
  fonttitle=\bfseries,
  left=1mm,
  right=1mm,
  top=1mm,
  bottom=1mm,
  middle=1mm
}

\begin{document}

\begin{frame}
\titlepage
\end{frame}

\begin{frame}
\tableofcontents
\end{frame}

\begin{section}{Introduction}
 
 
\end{section}


\begin{section}{Diagnosticablit\'e d'un automate}
\begin{frame}{Intuition du problème}
\begin{figure}[H]
	\centering
    	\begin{tikzpicture}
    	    \node[] (x) at (-1,0) {};
            \node[draw,circle,fill=lightgray] (0) at (0,0) {0};
            \node[draw,circle,fill=lightgray] (1) at (1.25,1) {1};
            \node[draw,circle,fill=lightgray] (2) at (2.25,1) {2};
            \node[draw,circle,fill=lightgray] (3) at (3.25,1) {3};
            \node[draw,circle,fill=lightgray] (4) at (1.25,-1) {4};
            \node[draw,circle,fill=lightgray] (5) at (2.25,-1) {5};
            \draw[->,>=latex] (x)->(0);
            \draw[->,>=latex, color=lightgray] (0) -> (1) node[midway,above]{$f$};
            \draw[->,>=latex] (1) -> (2) node[midway,above]{$a$};
            \draw[->,>=latex] (2) -> (3) node[midway,above]{$b$};
            \draw [->,>=latex] (3) edge[in=100,out=80,loop] node[above] {$a,b$} (3);
            \draw[->,>=latex] (0) -> (4) node[midway,above]{$a$};
            \draw[->,>=latex] (4) -> (5) node[midway,above]{$b$};
            \draw [->,>=latex] (5) edge[in=100,out=80,loop] node[above] {$a,b$} (5); 
        \end{tikzpicture}
 \caption{Système $A = (Q_A, \Sigma=\Sigma_o\sqcup \Sigma_{no}, \Delta_A, q_A, Q_A)$}    
\end{figure}
\pause
\begin{figure}
\begin{tikzpicture}
    	    \node[] (x) at (-1,0) {};
        	\node[draw,circle] (1) at (0,0) {0};
            \node[draw,circle,fill=lightgray](2) at (3,0){1};
            \draw[->,>=latex] (x)->(1);
            \draw[->,>=latex] (1)->(2) node[midway,above]{$f$};
            \draw [->,>=latex] (1) edge[in=100,out=80,loop] node[above] {$a,b$} (1);
            \draw [->,>=latex] (2) edge[in=100,out=80,loop] node[above] {$a,b,f$} (2);
        \end{tikzpicture}
\centering
 \caption{Langage de faute $L = (Q_L, \Sigma, \Delta_L, q_L, F_L)$}      
\end{figure}
\end{frame}
\begin{frame}{Diagnostiqueur}
\begin{center}
$D : \Sigma_o^* \to \{\mbox{Oui,\ Non,\ Peut-\^etre}\}$ 
\end{center}
\pause
Construction du diagnostiqueur :

  \begin{tabular}{cc}
 \begin{tikzpicture}[scale=0.9, transform shape]
	    \node[] (x) at (-1,0) {};
		\node[draw,circle] (0) at (0,0) {0,0};
        \node[draw,circle,fill=gray] (1) at (1.25,1) {1,1};
        \node[draw,circle,fill=gray] (2) at (2.75,1) {2,1};
        \node[draw,circle,fill=gray] (3) at (4.25,1) {3,1};
        \node[draw,circle] (4) at (1.25,-1) {4,0};
        \node[draw,circle] (5) at (2.75,-1) {5,0};
        \draw[->,>=latex] (x)->(0);
        \draw[->,>=latex, color=lightgray] (0) -> (1) node[midway,above] {$f$};
        \draw[->,>=latex] (1) -> (2) node[midway,above]{$a$};
        \draw[->,>=latex] (2) -> (3) node[midway,above]{$b$};
        \draw [->,>=latex] (3) edge[in=100,out=80,loop] node[above] {$a,b$} (3);
        \draw[->,>=latex] (0) -> (4) node[midway,above]{$a$};
        \draw[->,>=latex] (4) -> (5) node[midway,above]{$b$};
        \draw [->,>=latex] (5) edge[in=100,out=80,loop] node[above] {$a,b$} (5);
	\end{tikzpicture}
  & 
 \begin{tikzpicture}
	    \node[] (x) at (-1,0) {};
		\node[draw,circle] (0) at (0,0) {0};
        \node[draw,circle,fill=gray] (1) at (1.25,1) {1};
        \node[draw,circle,fill=gray] (2) at (2.75,1) {2};
        \node[draw,circle] (3) at (1.25,-1) {3};
        \node[draw,circle] (4) at (2.75,-1) {4};
        \draw[->,>=latex] (x)->(0);
        \draw[->,>=latex] (0) -> (1) node[midway,above]{$a$};
        \draw[->,>=latex] (1) -> (2) node[midway,above]{$b$};
        \draw [->,>=latex] (2) edge[in=100,out=80,loop] node[above] {$a,b$} (3);
        \draw[->,>=latex] (0) -> (3) node[midway,above]{$a$};
        \draw[->,>=latex] (3) -> (4) node[midway,above]{$b$};
        \draw [->,>=latex] (4) edge[in=100,out=80,loop] node[above] {$a,b$} (5);
	\end{tikzpicture}   \\
	\\
	$B=A \times L$  & $C=\operatorname{cl}_\varepsilon(B[\varepsilon/no])$ \\
  \end{tabular}
	
\end{frame}

\begin{frame}{Diagnostiqueur}
\begin{center}
$D : \Sigma_o^* \to \{\mbox{Oui,\ Non,\ Peut-\^etre}\}$ 
\end{center}

Construction du diagnostiqueur :

  \begin{tabular}{cc}
 \begin{tikzpicture}
	    \node[] (x) at (-1,0) {};
		\node[draw,circle] (0) at (0,0) {0};
        \node[draw,circle,fill=gray] (1) at (1.25,1) {1};
        \node[draw,circle,fill=gray] (2) at (2.75,1) {2};
        \node[draw,circle] (3) at (1.25,-1) {3};
        \node[draw,circle] (4) at (2.75,-1) {4};
        \draw[->,>=latex] (x)->(0);
        \draw[->,>=latex] (0) -> (1) node[midway,above]{$a$};
        \draw[->,>=latex] (1) -> (2) node[midway,above]{$b$};
        \draw [->,>=latex] (2) edge[in=100,out=80,loop] node[above] {$a,b$} (3);
        \draw[->,>=latex] (0) -> (3) node[midway,above]{$a$};
        \draw[->,>=latex] (3) -> (4) node[midway,above]{$b$};
        \draw [->,>=latex] (4) edge[in=100,out=80,loop] node[above] {$a,b$} (5);
	\end{tikzpicture}
  & 
   \begin{tikzpicture}[scale=0.8, transform shape]
   		\node[] (h) at (-2,-1.75) {};
	    \node[] (x) at (-1.3,0) {};
		\node[draw,circle, minimum size = 1.3cm] (0) at (0,0) {$\{0\}$};
        \node[draw,circle,fill=lightgray] (1) at (2,0) {$\{1,3\}$};
        \node[draw,circle,fill=lightgray] (2) at (4,0) {$\{2,4\}$};
         \draw[->,>=latex] (x)->(0);
         \draw[->,>=latex] (0) -> (1) node[midway,above]{$a$};
        \draw[->,>=latex] (1) -> (2) node[midway,above]{$b$};
        \draw [->,>=latex] (2) edge[in=100,out=80,loop] node[above] {$a,b$} (3);
        \end{tikzpicture}
      
   \\
	\\
	 $C=\operatorname{cl}_\varepsilon(B[\varepsilon/no])$  &  $D=\det'(C)$\\
  \end{tabular}
	
\end{frame}

\begin{frame}{Diagnostiqueur}

\begin{figure}
\centering
 \begin{tikzpicture}[scale=1, transform shape]
	    \node[] (x) at (-1.2,0) {};
		\node[draw,circle] (0) at (0,0) {Non};
        \node[draw,circle,fill=lightgray, minimum size = 1.1cm] (1) at (2,0) {?};
        \node[draw,circle,fill=lightgray, minimum size = 1.1cm] (2) at (4,0) {?};
         \draw[->,>=latex] (x)->(0);
         \draw[->,>=latex] (0) -> (1) node[midway,above]{$a$};
        \draw[->,>=latex] (1) -> (2) node[midway,above]{$b$};
        \draw [->,>=latex] (2) edge[in=100,out=80,loop] node[above] {$a,b$} (3);
        \end{tikzpicture}
  \caption{$D : \Sigma_o^* \to \{\mbox{Oui,\ Non,\ Peut-\^etre}\}$ }
\end{figure}
\pause
\textbf{Si une faute se produit, combien de temps faut il attendre au maximum avant de le savoir ?}\\
\pause
Longueur du plus long chemin constitué uniquement d'états de doute dans le diagnostiqueur. \\

\end{frame}

\begin{frame}{Définition formelle}
\centering
\begin{itemize}
\item On dit que $L$ est diagnosticable dans le système $A$ si :
$$\begin{array}{l}
\exists n \in \mathbb N, \forall w \in L,\\
\forall u \in \Sigma^*, \left(\left|\pi_{\Sigma_o}(u)\right|\ge n \land wu \in \mathcal L(A)\right) \implies\\
\forall v \in \mathcal L(A), \pi_{\Sigma_o}(v)=\pi_{\Sigma_o}(wu) \implies v \in L
\end{array}$$ 
\vspace{1cm}
\pause
\item On peut décider la diagnosticabilité en temps polynomial.
\end{itemize}
\end{frame}
\end{section}

\begin{section}{Opacit\'e et observabilit\'e}
  \begin{frame}{Système de transition d'états étiqueté}
    \begin{block}{D\'efinition : LTS}
      Un système de transition d'états étiqueté LTS est un quadruplet
      $$LTS = (S,L,\Delta,S_0)$$
      
      avec
	\begin{itemize}
	  \item $S$ : un ensemble d'\'etats (potentiellement infini)
	  \item $L$ : un ensemble d'étiquettes(potentiellement infini)
	  \item $\Delta \subseteq S\times L \times S$ : une relation de transition
	  \item $S_0$ : un ensemble fini non vide des \'etats initiaux 
	\end{itemize}
      
    \end{block}
  \end{frame}

  \begin{frame}{Fonction $Run$ d'un syst\`eme de transition d'\'etats \'etiquet\'e}
    \begin{block}{$Run$}
      Un $Run$ d'un LTS est un couple $(s_0,\lambda)$ avec :

	\begin{itemize}
	  \item $s_0 \in S_0$
	  \item $\lambda = l_1 \dots l_n$ : une s\'equence finie d'\'etiquettes telle qu'il existe $s_1,s_2,\dots ,s_n \in S$ tels que $\forall i \in \{1, \dots n\}, (s_{i-1},l_i,s_i) \in \Delta$ 
	\end{itemize}

	Notons $s_n$ par $s_0\oplus \lambda$.
    \end{block}
    
    \begin{block}{$Run$ d'un LTS}

	L'ensemble des tous les $Run$ est not\'e par $Run(LTS)$.

	Nous notons le langage g\'en\'er\'e par $LTS$ par 

	$$\mathcal{L}(LTS) = \{\lambda | \exists s_0 \in S_0 \mbox{ tel que } (s_0,\lambda) \in Run(LTS)\}$$
   
    \end{block}
  \end{frame}
  
  \begin{frame}{Fonction d'observation}
    \begin{block}{D\'efinition de fonction d'observation}
      $\Omega : Run(LTS) \rightarrow L_{obs}^*$
    \end{block}
    
    \begin{block}{Diff\'erentes capacit\'e de fonction d'observation}
     <<Fonction d'observation>> sur les mots compos\'es par les \'etiquettes $\Omega_{etiq} : L^* \rightarrow L_{obs}^{\varepsilon}$ telle que 
	
	$\forall (s,\lambda)= (s,l_1\dots l_n) \in Run(LTS), \Omega(s,\lambda) = \Omega_{etiq}(k_1)\circ\Omega_{etiq}(k_2)\circ\dots\circ\Omega_{etiq}(k_n)$

      \begin{itemize}
        \item Statique : $k_i = l_i$
	\item Dynamique : $k_i = l_1 \dots l_i$
	\item Orwellien : $k_i = \lambda$
	\item m-Orwellien : $k_i = l_{max(1,i-m+1)} \dots l_{min(n,i+m-1)}$

      \end{itemize}
    \end{block}   
   
  \end{frame}
  
  \begin{frame}{D\'efinition de l'opacit\'e}
    \begin{block}{D\'efinition}
      \begin{itemize}
       \item $\Omega$ : une fonction d'observation 
       \item $Secret$ un sous-ensemble de $Run(LTS)$
      \end{itemize}

     Nous disons que $Secret$ est opaque par rapport \`a $\Omega$ si et seulement si
	
	$$\forall (s,\lambda) \in Secret, \exists (s',\lambda')\not\in Secret \mbox{ tel que } \Omega(s,\lambda) = \Omega (s',\lambda')$$
	
	D\'efinition \'equivalente ensembliste:
	
	$$\Omega(Secret)\subseteq \Omega(Run(LTS)-Secret)$$
    \end{block}
  \end{frame}
  
  \begin{frame}{Exemple :}
    \begin{figure}[H]
                \centering
                \begin{tikzpicture}
                        \node[] (x) at (-1,0) {};
                        \node[draw,circle] (0) at (0,0) {0};
                        \node[draw,circle] (1) at (3,2) {1};
                        \node[draw,circle] (2) at (3,-2) {2};
                        \node[draw,circle] (3) at (6,1) {3};
                        \node[draw,circle] (4) at (6,3) {4};
                        \node[draw,circle] (5) at (6,-2) {5};
                        \draw[->,>=latex] (x)->(0);
                        \draw[->,>=latex,line width = 1mm] (0)->(1) node[midway,above]{$h$};
                        \draw[->,>=latex] (0)->(2) node[midway,above]{$p$};
                        \draw[->,>=latex] (1)->(3) node[midway,above]{$a$};
                        \draw[->,>=latex] (1)->(4) node[midway,above]{$b$};
                        \draw[->,>=latex] (2)->(5) node[midway,above]{$a$};
                \end{tikzpicture}
                \caption{$L = \{h,p,a,b\}$,$L_o = {a,b}$, Le langage que nous cherchons \`a rendre opaque est $L = L^*.h.L^*$}
		\end{figure}		
   
  \end{frame}

  
  \begin{frame}{Diff\'erents types d'opacit\'e}
    $Secret$ est $init-opaque/final-opaque/total-opaque$ s'il existe un sous-ensemble  $Secret'$ de l'ensemble des \'etats de $LTS$ tel que :
    
    $\forall (s,\lambda) \in Run(LTS), (s, \lambda) \in Secret \Leftrightarrow \sigma \in Secret'$
    
    \begin{block}{Diff\'erents types}
      \begin{itemize}
	\item $initial-opaque$ : $\sigma = s$
	\item $final-opaque$ : $\sigma = s\oplus \lambda$
	\item $total-opaque$ : $\sigma = s s_1 \dots s_n \mbox{ avec } s_i = s\oplus(l_1\dots l_i)$
      \end{itemize} 

    \end{block}


  \end{frame}

 
\end{section}


\begin{section}{Non-interf\'erence des syst\`eme et les r\'eductions vers les autres propri\'et\'es}


\begin{frame}{D\'efinition de la non-terf\'erence}
\begin{block}{D\'efinition}
\begin{itemize}
  \item $LTS = (S,L,\Delta,S_0)$ avec $L=L_{high}\sqcup L_{low}$
  \item $L_{high}$ : \'ev\`enements \`a cacher
  \item $L_{low}$ : \'ev\`enements observables
\end{itemize}

$LTS$ est dit non-intef\'erent si

$$\pi_{L_{low}}(\mathcal L(LTS)) \subseteq \mathcal L(LTS)$$

Intuition : $LTS$ est non-interf\'erent ssi un observateur n'observant que les \'ev\`enements \emph{low} ne peut pas d\'etecter la pr\'esence d'un \'ev\`enement \emph{high}
\end{block}
\end{frame}
 
\begin{frame}{R\'eduction : Opacit\'e $\to$ Non-interf\'erence}
\begin{beamerlikethm}{Th\'eor\`eme {\cite[page 5]{BryansKMR08}}}
Toute propri\'et\'e d'opacit\'e initiale reposant sur une fonction d'observation statique peut se r\'eduire \`a un probl\`eme  de non-interf\'erence.
\end{beamerlikethm}

Id\'ee de la preuve :

\begin{figure}[H]
  \centering
  \begin{tikzpicture}
    \node[] (x) at (-1,0) {};
    \node[draw,circle] (0) at (0,0) {$q_0$};
    \node[draw,circle] (1) at (2,0) {$q_1$};
    \node[draw,rectangle] (2) at (0,-2) {$A\setminus S$};
    \node[draw,rectangle] (3) at (2,-2) {$S$};
    \draw[->,>=latex] (x)->(0);
    \draw[->,>=latex] (0)->(1) node[midway,above]{$a$};
    \draw[->,>=latex] (0)->(2) node[midway,left]{$a$};
    \draw[->,>=latex] (1)->(3) node[midway,left]{$h$};
    
  \end{tikzpicture}
\end{figure}

$$A \subseteq A\setminus S \iff \pi_{low}(a(A\setminus S) + a h S)\subseteq a(A\setminus S) + a h S$$
\end{frame}

\begin{frame}{R\'eduction : Non-interf\'erence $\to$ Opacit\'e}
\begin{beamerlikethm}{Th\'eor\`eme {\cite[page 5]{BryansKMR08}}}
Tout probl\`eme de non-interf\'erence peut se r\'eduire \`a un probl\`eme d'opacit\'e reposant sur une fonction d'observation statique.
\end{beamerlikethm}

Id\'ee de la preuve :

$$S=\Sigma^*\Sigma_h\Sigma^*, \Omega = \pi_{low}$$

$$\pi_{low}(A) \subseteq A \iff \Omega(A) \subseteq \Omega(A \setminus S)$$
\end{frame}

\end{section}


\begin{section}{Conclusion}
 
\end{section}

\begin{frame}{Bibliographie}
\bibliography{../rapport/bibliography}
\end{frame}
\end{document}
