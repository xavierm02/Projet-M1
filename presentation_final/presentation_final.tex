\documentclass[11pt]{beamer}
\usetheme{Singapore}
\usepackage[utf8]{inputenc}
\usepackage[francais]{babel}
\usepackage[T1]{fontenc}
\usepackage{amsmath}
\usepackage{amsfonts}
\usepackage{amssymb}
\usepackage{tikz}
\usepackage{float}


\newtheorem{mydef}{D\'efinition}

\author{François Godi \and Xavier Montillet \and Chen Qian}
\title{Opacit\'e des automates}
\setbeamercovered{transparent} 
\setbeamertemplate{navigation symbols}{} 
%\logo{} 
%\institute{} 
%\date{} 
%\subject{} 
\usepackage{tcolorbox}
\tcbuselibrary{skins}

\colorlet{xlightblue}{blue!5}

\newtcolorbox{beamerlikethm}[1]{
  title=#1,
  beamer,
  colback=xlightblue,
  colframe=blue!30,
  fonttitle=\bfseries,
  left=1mm,
  right=1mm,
  top=1mm,
  bottom=1mm,
  middle=1mm
}

\begin{document}

\begin{frame}
\titlepage
\end{frame}

\begin{frame}
\tableofcontents
\end{frame}

\begin{section}{T-diagnosticabilit\'e}

  \begin{frame}{T-diagnosticabilit\'e}
    \begin{block}{Intuition}
      Il existe une borne $n$ telle que, pour toute faute $u$, si l'on a attendu $n$ nouveaux observables ($v$) après que $u$ ce soit produite, alors il y a eu un moment où l'on avait observé $\pi_{\Sigma_o}(uv')$ et où tous les autres mots qui produisent l'observation $\pi_{\Sigma_o}(uv')$ contiennent une faute. Ainsi, lorsque une faute a eu lieu, on peut toujours affirmer sans se tromper qu'elle eu lieu (au plus tard $n$ observables après qu'elle se soit produite).
    \end{block}
  \end{frame}

  \begin{frame}{D\'efinition formelle}
    Formellement nous pouvons d\'efinir la T-diagnosticabilit\'e comme ci-dessous
    
    \begin{mydef}{T-diagnosticabilité}
      
      $$
      \begin{array}{l}
        \exists n \in \mathbb{N}, \ \forall u \in L^{MIN}_F(S), \forall v \in \Sigma^*.\\
        (uv \in L(S) \ \wedge \  ||v||_{\Sigma_o}\geq n) \Rightarrow \ \exists v' \leq v. \\
(u.Pref(v') \subseteq L_F(S) \ \wedge \  [uv'] \cap L(S) \subseteq L_F(S))
      \end{array}
      $$

    \end{mydef}
    
  \end{frame}

  \begin{frame}{Example}
    
    \begin{figure}[H]
      \begin{center}
        \scalebox{1.5}{
          \begin{tikzpicture}
            \node[] (0) at (-1,0) {};
            \node[draw,circle] (1) at (0,0) {O};
            \node[draw,rectangle] (2) at (1.5,1) {N};
            \node[draw,rectangle] (3) at (3,0) {E};
            \node[draw,circle] (4) at (1.5,-1) {S};
            \draw[->,>=latex] (0)->(1) {}; 
            \draw[->,>=latex] (1)->(2) node[midway,above]{$a$};
            \draw[->,>=latex] (2)->(3) node[midway,above]{$a,b$};
            \draw[->,>=latex] (3)->(4) node[midway,below]{$b$};
            \draw[->,>=latex] (4)->(1) node[midway,below]{$a,b$};
            \draw [->,>=latex] (1) edge[in=100,out=80,loop] node[above] {$a$} (1);
            \draw [->,>=latex] (3) edge[in=100,out=80,loop] node[above] {$b$} (3);
            \draw [->,>=latex,dashed] (1) edge [in=110,out=0] node[midway,above]{$b$} (4) ;
          \end{tikzpicture}
        }
      \end{center}
      \caption{V\'erification de T-diagnosticabilit\'e. Les \'etats carr\'es repr\'esentent les \'etats fautifs et les cercles repr\'esentent les \'etats non-fautifs}
    \end{figure}

  \end{frame}

  \begin{frame}{Non-T-diagnosticabilit\'e}
    \begin{figure}[H]
      \caption{Deux ex\'ecutions qui montrent que le syst\`eme est non-T-diagnosticable}
      \begin{center}
        \begin{tikzpicture}
          \node[](0) at (-1,0) {};
          \node[draw,circle] (1) at (0,0) {O};
          \node[draw,rectangle] (2) at (2,0) {N};
          \node[draw,rectangle] (3) at (4,0) {E};
          \node[draw,circle] (4) at (6,0) {S};
          \draw[->,>=latex] (0)->(1) {};
          \draw[->,>=latex] (1)->(2) node[midway,above]{$a$};
          \draw[->,>=latex] (2)->(3) node[midway,above]{$a$};
          \draw[->,>=latex] (3)->(4) node[midway,above]{$b$};
          
          
          \node[](5) at (-1,-2) {};
          \node[draw,circle] (6) at (0,-2) {O};
          \node[draw,circle] (7) at (2,-2) {O};
          \node[draw,circle] (8) at (4,-2) {O};
          \node[draw,circle] (9) at (6,-2) {S};
          
          \draw[->,>=latex] (5)->(6) {};
          \draw[->,>=latex] (6)->(7) node[midway,above]{$a$};
          \draw[->,>=latex] (7)->(8) node[midway,above]{$a$};
          \draw[->,>=latex] (8)->(9) node[midway,above]{$b$};
        \end{tikzpicture}
      \end{center}
    \end{figure}
  \end{frame}
  \begin{frame}{T-Diagnostiqueur}
    La construction du T-diagnostiqueur consiste seulement à d\'eterminiser l'automate $S$. 
  \end{frame}

  \begin{frame}{Relation T-diagnosticabilit\'e et diagnosticabilit\'e}
    Reprenons les deux d\'efinition :
    \begin{itemize}
    \item $$\begin{array}{l}
      \exists n \in \mathbb N, \forall u \in L_F(S),\\
      \forall v \in \Sigma^*, \left(||v||_{\Sigma_o} \ge n \land uv \in  L(S)\right) \implies\\
      \forall w \in L(S), \pi_{\Sigma_o}(w)=\pi_{\Sigma_o}(uv) \implies w \in L_F(S)
    \end{array}$$
    \item $$\begin{array}{l}
      \exists n \in \mathbb{N}, \ \forall u \in L^{MIN}_F(S), \forall v \in \Sigma^*.\\
      (uv \in L(S) \ \wedge \  ||v||_{\Sigma_o}\geq n) \Rightarrow \ \exists v' \leq v. \\
      (u.Pref(v') \subseteq L_F(S) \ \wedge \  [uv'] \cap L(S) \subseteq L_F(S))
    \end{array}$$

    \end{itemize}

    La T-diagnosticabilité généralise la diagnosticabilité

  \end{frame}
\end{section}

\end{document}
