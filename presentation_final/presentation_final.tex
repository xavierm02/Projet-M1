\documentclass[11pt]{beamer}
\usetheme{Singapore}
\usepackage[utf8]{inputenc}
\usepackage[francais]{babel}
\usepackage[T1]{fontenc}
\usepackage{amsmath}
\usepackage{amsfonts}
\usepackage{amssymb}
\usepackage{tikz}
\usepackage{float}

\usetikzlibrary{arrows,positioning,automata,shadows}


\newtheorem{mydef}{D\'efinition}
\newtheorem{myth}{Th\'eor\`eme}
\newtheorem{myco}{Corollaire}
\newtheorem{myconj}{Conjecture}
\newtheorem{mydis}{Discussion}
\newtheorem{myid}{Idée}

\usepackage{wasysym}
\newcommand{\es}{\square}
\newcommand{\fs}{\blacksquare}
\newcommand{\ec}{\ocircle}

\author{François Godi \and Xavier Montillet \and Chen Qian}
\title{Opacit\'e des automates}
\setbeamercovered{transparent} 
\setbeamertemplate{navigation symbols}{} 
%\logo{} 
%\institute{} 
%\date{} 
%\subject{} 
\usepackage{tcolorbox}
\tcbuselibrary{skins}

\colorlet{xlightblue}{blue!5}

\newtcolorbox{beamerlikethm}[1]{
  title=#1,
  beamer,
  colback=xlightblue,
  colframe=blue!30,
  fonttitle=\bfseries,
  left=1mm,
  right=1mm,
  top=1mm,
  bottom=1mm,
  middle=1mm
}

\begin{document}

\begin{frame}
\titlepage
\end{frame}

\section{Introduction}

\subsection{Contexte du problème}
\begin{frame}{Système à évènements discrets avec fautes}
\begin{figure}[H]
  \begin{center}
  \scalebox{1.5}{
    \begin{tikzpicture}
      \node[] (0) at (-1,0) {};
      \node[draw,circle] (1) at (0,0) {1};
      \node[draw,rectangle] (2) at (2,0) {2};
      \draw[->,>=latex] (0)->(1) {}; 
      \draw[->,>=latex] (1)->(2) node[midway,above]{$a$};
     
      \draw [->,>=latex] (1) edge[in=100,out=80,loop] node[above] {$a$} (1);
       \draw [->,>=latex] (2) edge[in=100,out=80,loop] node[above] {$b$} (2);
    
    \end{tikzpicture}
    }
  \end{center}
    \caption{Automate représentant un système à évènements discrets avec fautes. Les carrés représentent}
\end{figure}
\end{frame}

\begin{frame}{Diagnosticabilité}

\begin{figure}[H]
  \begin{center}
    \begin{tikzpicture}
      \node[] (0) at (-1,0) {};
      \node[draw,circle] (1) at (0,0) {1};
      \node[draw,rectangle] (2) at (2,0) {2};
      \draw[->,>=latex] (0)->(1) {}; 
      \draw[->,>=latex] (1)->(2) node[midway,above]{$a$};
     
      \draw [->,>=latex] (1) edge[in=100,out=80,loop] node[above] {$a$} (1);
       \draw [->,>=latex] (2) edge[in=100,out=80,loop] node[above] {$b$} (2);
    
    \end{tikzpicture}
    
  \end{center}
\end{figure}

On dit que $S$ est diagnosticable \cite{SamSRST96} si et seulement si :

$$\begin{array}{l}
\exists n_0 \in \mathbb N, \forall x \in L_F(S),\\
\forall y \in \Sigma^*, \left(||y||_{\Sigma_o} \ge n_0 \land xy \in  L(S)\right) \implies\\
\forall z \in L(S), \pi_{\Sigma_o}(z)=\pi_{\Sigma_o}(xy) \implies z \in L_F(S)
\end{array}$$

\end{frame}
\subsection{Contribution}
\begin{frame}{Fautes réparables}
\begin{figure}[H]
  \begin{center}
  \scalebox{1.5}{
    \begin{tikzpicture}
      \node[] (0) at (-1,0) {};
      \node[draw,circle] (1) at (0,0) {O};
      \node[draw,rectangle] (2) at (1.5,1) {N};
      \node[draw,rectangle] (3) at (3,0) {E};
      \node[draw,circle] (4) at (1.5,-1) {S};
      \draw[->,>=latex] (0)->(1) {}; 
      \draw[->,>=latex] (1)->(2) node[midway,above]{$a$};
      \draw[->,>=latex] (2)->(3) node[midway,above]{$a,b$};
      \draw[->,>=latex] (3)->(4) node[midway,below]{$b$};
      \draw[->,>=latex] (4)->(1) node[midway,below]{$a,b$};
      \draw [->,>=latex] (1) edge[in=100,out=80,loop] node[above] {$a$} (1);
      \draw [->,>=latex] (3) edge[in=100,out=80,loop] node[above] {$b$} (3);
      \draw [->,>=latex,dashed] (1) edge [in=110,out=0] node[midway,above]{$b$} (4) ;
    \end{tikzpicture}
    }
  \end{center}
    \caption{V\'erification de T-diagnosticabilit\'e. Les \'etats carr\'es repr\'esentent les \'etats fautifs et les cercles repr\'esentent les \'etats non-fautifs}
\end{figure}
\end{frame}
\begin{frame}
\tableofcontents
\end{frame}

\begin{section}{T-diagnosticabilit\'e}

  \begin{frame}{T-diagnosticabilit\'e}
    \begin{block}{Intuition}
      Il existe une borne $n$ telle que, pour toute faute $u$, si l'on a attendu $n$ nouveaux observables ($v$) après que $u$ ce soit produite, alors il y a eu un moment où l'on avait observé $\pi_{\Sigma_o}(uv')$ et où tous les autres mots qui produisent l'observation $\pi_{\Sigma_o}(uv')$ contiennent une faute. Ainsi, lorsque une faute a eu lieu, on peut toujours affirmer sans se tromper qu'elle eu lieu (au plus tard $n$ observables après qu'elle se soit produite).
    \end{block}
  \end{frame}

  \begin{frame}{D\'efinition formelle}
    Formellement nous pouvons d\'efinir la T-diagnosticabilit\'e comme ci-dessous

    \begin{mydef}[Langage de faute minimal]

      $$L_{F}^{MIN}(S) =  \{v.a \in L_F(S) \mid v\not \in L_F(S), a\in \Sigma \} $$
    \end{mydef}
    
    \begin{mydef}{T-diagnosticabilité}
      
      $$
      \begin{array}{l}
        \exists n \in \mathbb{N}, \ \forall u \in L^{MIN}_F(S), \forall v \in \Sigma^*.\\
        (uv \in L(S) \ \wedge \  ||v||_{\Sigma_o}\geq n) \Rightarrow \ \exists v' \leq v. \\
(u.Pref(v') \subseteq L_F(S) \ \wedge \  [uv'] \cap L(S) \subseteq L_F(S))
      \end{array}
      $$

    \end{mydef}
    
  \end{frame}

  \begin{frame}{Example}
    
    \begin{figure}[H]
      \begin{center}
        \resizebox{3.0cm}{!}{
          \begin{tikzpicture}
            \node[] (0) at (-1,0) {};
            \node[draw,circle] (1) at (0,0) {O};
            \node[draw,rectangle] (2) at (1.5,1) {N};
            \node[draw,rectangle] (3) at (3,0) {E};
            \node[draw,circle] (4) at (1.5,-1) {S};
            \draw[->,>=latex] (0)->(1) {}; 
            \draw[->,>=latex] (1)->(2) node[midway,above]{$a$};
            \draw[->,>=latex] (2)->(3) node[midway,above]{$a,b$};
            \draw[->,>=latex] (3)->(4) node[midway,below]{$b$};
            \draw[->,>=latex] (4)->(1) node[midway,below]{$a,b$};
            \draw [->,>=latex] (1) edge[in=100,out=80,loop] node[above] {$a$} (1);
            \draw [->,>=latex] (3) edge[in=100,out=80,loop] node[above] {$b$} (3);
            \draw [->,>=latex,dashed] (1) edge [in=110,out=0] node[midway,above]{$b$} (4) ;
          \end{tikzpicture}
          }
      \end{center}
      \caption{V\'erification de T-diagnosticabilit\'e. Les \'etats carr\'es repr\'esentent les \'etats fautifs et les cercles repr\'esentent les \'etats non-fautifs}
    \end{figure}


    \begin{figure}[H]
      \begin{center}
        \resizebox{3.0cm}{!}{
        \begin{tikzpicture}
          \node[](0) at (-1,0) {};
          \node[draw,circle] (1) at (0,0) {O};
          \node[draw,rectangle] (2) at (2,0) {N};
          \node[draw,rectangle] (3) at (4,0) {E};
          \node[draw,circle] (4) at (6,0) {S};
          \draw[->,>=latex] (0)->(1) {};
          \draw[->,>=latex] (1)->(2) node[midway,above]{$a$};
          \draw[->,>=latex] (2)->(3) node[midway,above]{$a$};
          \draw[->,>=latex] (3)->(4) node[midway,above]{$b$};
          
          
          \node[](5) at (-1,-2) {};
          \node[draw,circle] (6) at (0,-2) {O};
          \node[draw,circle] (7) at (2,-2) {O};
          \node[draw,circle] (8) at (4,-2) {O};
          \node[draw,circle] (9) at (6,-2) {S};
          
          \draw[->,>=latex] (5)->(6) {};
          \draw[->,>=latex] (6)->(7) node[midway,above]{$a$};
          \draw[->,>=latex] (7)->(8) node[midway,above]{$a$};
          \draw[->,>=latex] (8)->(9) node[midway,above]{$b$};
        \end{tikzpicture}}
      \end{center}
      \caption{Deux ex\'ecutions qui montrent que le syst\`eme est non-T-diagnosticable}
    \end{figure}
  \end{frame}
  \begin{frame}{T-Diagnostiqueur}
    La construction du T-diagnostiqueur consiste seulement à d\'eterminiser l'automate $S$. 
  \end{frame}

  \begin{frame}{Relation T-diagnosticabilit\'e et diagnosticabilit\'e}
    Reprenons les deux d\'efinition :
    \begin{itemize}
    \item $$\begin{array}{l}
      \exists n \in \mathbb N, \forall u \in L_F(S),\\
      \forall v \in \Sigma^*, \left(||v||_{\Sigma_o} \ge n \land uv \in  L(S)\right) \implies\\
      \forall w \in L(S), \pi_{\Sigma_o}(w)=\pi_{\Sigma_o}(uv) \implies w \in L_F(S)
    \end{array}$$
    \item $$\begin{array}{l}
      \exists n \in \mathbb{N}, \ \forall u \in L^{MIN}_F(S), \forall v \in \Sigma^*.\\
      (uv \in L(S) \ \wedge \  ||v||_{\Sigma_o}\geq n) \Rightarrow \ \exists v' \leq v. \\
      (u.Pref(v') \subseteq L_F(S) \ \wedge \  [uv'] \cap L(S) \subseteq L_F(S))
    \end{array}$$

    \end{itemize}

    La T-diagnosticabilité généralise la diagnosticabilité

  \end{frame}
\end{section}

\begin{section}{Décision et complexité}

\begin{frame}{Twin-machine}
\begin{figure}
\label{a2diag}
\scalebox{0.7}{
\begin{tikzpicture}[->,>=stealth',shorten >=1pt,auto,node distance=3cm,
  thick,main node/.style={circle,draw,font=\sffamily\Large\bfseries}, align=center]

  \baselineskip=0.1cm
  \lineskiplimit=4cm
  \lineskip=0.1cm

  \node[main node] (4) [fill=green!20] {$\fs,\fs$};
  \node[main node] (2) [below left of=4, fill=orange!20] {$\es, \ec$};
  \node[main node] (1) [below right of=2, fill=blue!20, initial below] {$\ec, \ec$}; %il faudrait enlever le start
  \node[main node] (3) [below right of=4, fill=orange!20] {$\ec,\es$};
  \node[main node] (6) [right of=4, fill=green!20] {$\fs, \ec$};
  \node[main node] (5) [left of=4, fill=green!20] {$\ec, \fs$};

  \path[every node/.style={font=\sffamily\small}]
    (1) edge [bend left=30]  		(4)
        edge [bend left=30]  		(2)
        edge [bend right=20] 		(3)
        edge [loop above]    		(1)
    (2) edge [loop left, dashed]    	(2)
        edge [bend left=15]  		(4)
    (3) edge [loop right, dashed]   	(3)
        edge [bend right=15] 		(4)
    (4) edge [bend left=30]  		(1)
        edge [loop above]    		(4)
        edge [bend left=10]  		(5)
        edge [bend left=10]  		(6)
    (5) edge [loop left]     		(5)
        edge [bend right=75] 		(1)
        edge [bend left=10]  		(4)
        edge                 		(2)
    (6) edge [loop right=20] 		(6)
        edge [bend left=10]  		(4)
        edge [bend left=75]  		(1)
        edge                 		(3);
\end{tikzpicture}
}
\caption{Description des paires d'excécutions compatibles avec la T-diagnosticabilité. Les carrés sont coloriés lorsque la faute est détectée.}
\end{figure}
\end{frame}

\begin{frame}{$T_2$-diagnosticabilité $\not\Rightarrow$ T-diagnosticabilité}
\begin{figure}

\label{t2diag}
$D_{1,3}$ :
\scalebox{0.7}{
\begin{tikzpicture}[->,>=stealth',shorten >=1pt,auto,node distance=1cm,
  thick,main node/.style={circle,draw,font=\sffamily\bfseries}, align=center]

  \baselineskip=0.1cm
  \lineskiplimit=4cm
  \lineskip=0.1cm

  \node[main node, initial, initial text={}] (0) {0};
  \node[main node, shape = rectangle, right of = 0] (1) {1};
  \node[main node, shape = rectangle, right of = 1] (2) {2};
  \node[main node, shape = rectangle, right of = 2] (3) {3};
  \node[main node, right of = 3] (4) {4};
  \node[main node, right of = 4] (5) {5};
  \node[main node, right of = 5] (6) {6};
  \node[main node, shape = rectangle, right of = 6] (7) {7};

  \path[every node/.style={font=\sffamily\small}]
    (0) edge (1)
    (1) edge (2)
    (2) edge (3)
    (3) edge (4)
    (4) edge (5)
    (5) edge (6)
    (6) edge (7)
    (7) edge [bend right = 30] (2);
\end{tikzpicture}
}
\newline
$D_{2,3}$ :
\scalebox{0.7}{
\begin{tikzpicture}[->,>=stealth',shorten >=1pt,auto,node distance=1cm,
  thick,main node/.style={circle,draw,font=\sffamily\bfseries}, align=center]

  \baselineskip=0.1cm
  \lineskiplimit=4cm
  \lineskip=0.1cm

  \node[main node, initial, initial text={}] (0) {0};
  \node[main node, shape = rectangle, right of = 0] (1) {1};
  \node[main node, right of = 1] (2) {2};
  \node[main node, shape = rectangle, right of = 2] (3) {3};
  \node[main node, shape = rectangle, right of = 3] (4) {4};
  \node[main node, shape = rectangle, right of = 4] (5) {5};
  \node[main node, right of = 5] (6) {6};
  \node[main node, right of = 6] (7) {7};

  \path[every node/.style={font=\sffamily\small}]
    (0) edge (1)
    (1) edge (2)
    (2) edge (3)
    (3) edge (4)
    (4) edge (5)
    (5) edge (6)
    (6) edge (7)
    (7) edge [bend right = 30] (2);
\end{tikzpicture}
}
\newline
$D_{3,3}$ : 
\scalebox{0.7}{
\begin{tikzpicture}[->,>=stealth',shorten >=1pt,auto,node distance=1cm,
  thick,main node/.style={circle,draw,font=\sffamily\bfseries}, align=center]

  \baselineskip=0.1cm
  \lineskiplimit=4cm
  \lineskip=0.1cm

  \node[main node, initial, initial text={}] (0) {0};
  \node[main node, shape = rectangle, right of = 0] (1) {1};
  \node[main node, right of = 1] (2) {2};
  \node[main node, right of = 2] (3) {3};
  \node[main node, right of = 3] (4) {4};
  \node[main node, shape = rectangle, right of = 4] (5) {5};
  \node[main node, shape = rectangle, right of = 5] (6) {6};
  \node[main node, shape = rectangle, right of = 6] (7) {7};

  \path[every node/.style={font=\sffamily\small}]
    (0) edge (1)
    (1) edge (2)
    (2) edge (3)
    (3) edge (4)
    (4) edge (5)
    (5) edge (6)
    (6) edge (7)
    (7) edge [bend right = 30] (2);
\end{tikzpicture}
}
\newline
\caption{Exemple d'automate $T_2$-diagnosticable mais non $T_3$-diagnosticable (et donc non $T$-diagnosticable)}
\end{figure}
\end{frame}

\iffalse
\begin{frame}{Algorithme}
La seule difficultée est de différentier les $\es$ des $\fs$ :
\begin{itemize}
	\item Pour chaque état $q$ de $A$, on calcule l'ensemble des états que l'on peut atteindre avec une exécution équivalente, c'est-à-dire l'ensemble $E_q = \{q' \mid \exists u, q_0 \overset{u}{\to}q \land q_0 \overset{u}{\to}q' \}$. Pour ce faire, on calcule le point fixe avec $q'\in E_q \land q \overset{x}{\to} p \land q' \overset{x}{\to} p' \implies p' \in E_p$, initialisé par $q\in E_q$. On marque ensuite comme $\fs$ les états $q$ fautifs tels que $E_q$ ne contient que des états fautifs. Ces états correspondent aux états pouvant être la composante gauche de $(\fs, \{\es\})$.

	\item On colorie ensuite les $\es$ pouvant être la composante gauche de $(\fs,\{\ec, \es\})$. Pour ce faire, on calcule le point fixe de la règle suivante : un $\es$ dont tous les prédécesseurs sont des $\fs$ devient un $\fs$.
\end{itemize}
On sait donc décider si un état fautif peut être un $\es$ ou non. L'alorithme ci-dessus est PSPACE.
\end{frame}

\begin{frame}{Algorithme (suite)}
\begin{itemize}
	\item Pour vérifier si une transition interdite est utilisée, on itère sur toutes les paires d'états de $A\times \text{det}(A)$ (sans construire $\text{det}(A)$ exlicitement) et s'il y a une transition entre les deux, on vérifie si elle est valide, si nécessaire un utilisant l'agorithme précédent pour savoir si l'on peut utiliser les transitions qui sont interdites depuis $(\es,\{\ec,\es\})$.
	\item Pour vérifier l'absence de cycles de $(\es,\{\ec,\es\})$
\end{itemize}
\end{frame}
\fi



\begin{frame}{Algorithme}
On construit $A\times \text{det}(A) \times B$ où B est l'automate ci-dessous complété par les transitions interdites. On vérifie ensuite que les transitions interdites ne sont en fait pas utilisées et qu'il n'y a pas de cycle dont la composante de droite reste dans l'état $(\es,\{\ec, \es\})$. Cet algorithme est EXPTIME.

\begin{myth}
La décision de la T-diagnosticabilité est EXPTIME.
\end{myth}

Nous pensons qu'il est possible d'éviter de construire explicitement le déterminisé afin d'obtenir un algorithme PSPACE.
\end{frame}

\begin{frame}{PSPACE-dur}
\begin{figure}
\label{t2diag}
$D_{1,3}$ : 
\scalebox{0.95}{
\begin{tikzpicture}[->,>=stealth',shorten >=1pt,auto,node distance=1cm,
  thick,main node/.style={circle,draw,font=\sffamily\bfseries}, align=center]

  \baselineskip=0.1cm
  \lineskiplimit=4cm
  \lineskip=0.1cm

  \node[main node, initial, initial text={}] (0) {0};
  \node[main node, shape = rectangle, right of = 0] (1) {1};
  \node[main node, shape = rectangle, right of = 1] (2) {2};
  \node[main node, shape = rectangle, right of = 2] (3) {3};
  \node[main node, right of = 3] (4) {4};
  \node[main node, right of = 4] (5) {5};
  \node[main node, right of = 5] (6) {6};
  \node[main node, shape = rectangle, right of = 6] (7) {7};

  \path[every node/.style={font=\sffamily\small}]
    (0) edge (1)
    (1) edge (2)
    (2) edge (3)
    (3) edge (4)
    (4) edge (5)
    (5) edge (6)
    (6) edge (7)
    (7) edge [bend right = 30] (2);
\end{tikzpicture}
}\newline
$D_{2,3}$ : 
\scalebox{0.95}{
\begin{tikzpicture}[->,>=stealth',shorten >=1pt,auto,node distance=1cm,
  thick,main node/.style={circle,draw,font=\sffamily\bfseries}, align=center]

  \baselineskip=0.1cm
  \lineskiplimit=4cm
  \lineskip=0.1cm

  \node[main node, initial, initial text={}] (0) {0};
  \node[main node, shape = rectangle, right of = 0] (1) {1};
  \node[main node, right of = 1] (2) {2};
  \node[main node, shape = rectangle, right of = 2] (3) {3};
  \node[main node, shape = rectangle, right of = 3] (4) {4};
  \node[main node, shape = rectangle, right of = 4] (5) {5};
  \node[main node, right of = 5] (6) {6};
  \node[main node, right of = 6] (7) {7};

  \path[every node/.style={font=\sffamily\small}]
    (0) edge (1)
    (1) edge (2)
    (2) edge (3)
    (3) edge (4)
    (4) edge (5)
    (5) edge (6)
    (6) edge (7)
    (7) edge [bend right = 30] (2);
\end{tikzpicture}
}\newline
$D_{3,3}$ : 
\scalebox{0.95}{
\begin{tikzpicture}[->,>=stealth',shorten >=1pt,auto,node distance=1cm,
  thick,main node/.style={circle,draw,font=\sffamily\bfseries}, align=center]

  \baselineskip=0.1cm
  \lineskiplimit=4cm
  \lineskip=0.1cm

  \node[main node, initial, initial text={}] (0) {0};
  \node[main node, shape = rectangle, right of = 0] (1) {1};
  \node[main node, right of = 1] (2) {2};
  \node[main node, right of = 2] (3) {3};
  \node[main node, right of = 3] (4) {4};
  \node[main node, shape = rectangle, right of = 4] (5) {5};
  \node[main node, shape = rectangle, right of = 5] (6) {6};
  \node[main node, shape = rectangle, right of = 6] (7) {7};

  \path[every node/.style={font=\sffamily\small}]
    (0) edge (1)
    (1) edge (2)
    (2) edge (3)
    (3) edge (4)
    (4) edge (5)
    (5) edge (6)
    (6) edge (7)
    (7) edge [bend right = 30] (2);
\end{tikzpicture}
}\newline
\end{figure}
\end{frame}


\begin{frame}{PSPACE-dur (2)}
\begin{figure}
\label{t3diag}
$D_{1,4}$ :
\scalebox{0.9}{
 \begin{tikzpicture}[->,>=stealth',shorten >=1pt,auto,node distance=0.8cm,
  thick,main node/.style={circle,draw,font=\sffamily\bfseries}, align=center]

  \baselineskip=0.1cm
  \lineskiplimit=4cm
  \lineskip=0.1cm

  \node[main node, initial, initial text={}] (0) {0};
  \node[main node, shape = rectangle, right of = 0] (1) {1};
  \node[main node, shape = rectangle, right of = 1] (2) {2};
  \node[main node, shape = rectangle, right of = 2] (3) {3};
  \node[main node, shape = rectangle, right of = 3] (4) {4};
  \node[main node, shape = rectangle, right of = 4] (5) {5};
  \node[main node, right of = 5] (6) {6};
  \node[main node, right of = 6] (7) {7};
  \node[main node, right of = 7] (8) {8};
  \node[main node, shape = rectangle, right of = 8] (9) {9};

  \path[every node/.style={font=\sffamily\small}]
    (0) edge (1)
    (1) edge (2)
    (2) edge (3)
    (3) edge (4)
    (4) edge (5)
    (5) edge (6)
    (6) edge (7)
    (7) edge (8)
    (8) edge (9)
    (9) edge [bend right = 30] (2);
\end{tikzpicture}
}\newline
$D_{2,4}$ :
\scalebox{0.9}{
 \begin{tikzpicture}[->,>=stealth',shorten >=1pt,auto,node distance=0.8cm,
  thick,main node/.style={circle,draw,font=\sffamily\bfseries}, align=center]

  \baselineskip=0.1cm
  \lineskiplimit=4cm
  \lineskip=0.1cm

  \node[main node, initial, initial text={}] (0) {0};
  \node[main node, shape = rectangle, right of = 0] (1) {1};
  \node[main node, right of = 1] (2) {2};
  \node[main node, shape = rectangle, right of = 2] (3) {3};
  \node[main node, shape = rectangle, right of = 3] (4) {4};
  \node[main node, shape = rectangle, right of = 4] (5) {5};
  \node[main node, shape = rectangle, right of = 5] (6) {6};
  \node[main node, shape = rectangle, right of = 6] (7) {7};
  \node[main node, right of = 7] (8) {8};
  \node[main node, right of = 8] (9) {9};

  \path[every node/.style={font=\sffamily\small}]
    (0) edge (1)
    (1) edge (2)
    (2) edge (3)
    (3) edge (4)
    (4) edge (5)
    (5) edge (6)
    (6) edge (7)
    (7) edge (8)
    (8) edge (9)
    (9) edge [bend right = 30] (2);
\end{tikzpicture}
}\newline
$D_{3,4}$ :
\scalebox{0.9}{
 \begin{tikzpicture}[->,>=stealth',shorten >=1pt,auto,node distance=0.8cm,
  thick,main node/.style={circle,draw,font=\sffamily\bfseries}, align=center]

  \baselineskip=0.1cm
  \lineskiplimit=4cm
  \lineskip=0.1cm

  \node[main node, initial, initial text={}] (0) {0};
  \node[main node, shape = rectangle, right of = 0] (1) {1};
  \node[main node, right of = 1] (2) {2};
  \node[main node, right of = 2] (3) {3};
  \node[main node, right of = 3] (4) {4};
  \node[main node, shape = rectangle, right of = 4] (5) {5};
  \node[main node, shape = rectangle, right of = 5] (6) {6};
  \node[main node, shape = rectangle, right of = 6] (7) {7};
  \node[main node, shape = rectangle, right of = 7] (8) {8};
  \node[main node, shape = rectangle, right of = 8] (9) {9};

  \path[every node/.style={font=\sffamily\small}]
    (0) edge (1)
    (1) edge (2)
    (2) edge (3)
    (3) edge (4)
    (4) edge (5)
    (5) edge (6)
    (6) edge (7)
    (7) edge (8)
    (8) edge (9)
    (9) edge [bend right = 30] (2);
\end{tikzpicture}
}\newline
$D_{3,4}$ : 
\scalebox{0.9}{
\begin{tikzpicture}[->,>=stealth',shorten >=1pt,auto,node distance=0.8cm,
  thick,main node/.style={circle,draw,font=\sffamily\bfseries}, align=center]

  \baselineskip=0.1cm
  \lineskiplimit=4cm
  \lineskip=0.1cm

  \node[main node, initial, initial text={}] (0) {0};
  \node[main node, shape = rectangle, right of = 0] (1) {1};
  \node[main node, shape = rectangle, right of = 1] (2) {2};
  \node[main node, shape = rectangle, right of = 2] (3) {3};
  \node[main node, right of = 3] (4) {4};
  \node[main node, right of = 4] (5) {5};
  \node[main node, right of = 5] (6) {6};
  \node[main node, shape = rectangle, right of = 6] (7) {7};
  \node[main node, shape = rectangle, right of = 7] (8) {8};
  \node[main node, shape = rectangle, right of = 8] (9) {9};

  \path[every node/.style={font=\sffamily\small}]
    (0) edge (1)
    (1) edge (2)
    (2) edge (3)
    (3) edge (4)
    (4) edge (5)
    (5) edge (6)
    (6) edge (7)
    (7) edge (8)
    (8) edge (9)
    (9) edge [bend right = 30] (2);
\end{tikzpicture}
}\newline
\end{figure}

\end{frame}

\begin{frame}{PSPACE-dur (3)}
\begin{tikzpicture}[->,>=stealth',shorten >=1pt,auto,node distance=2cm,
  thick,main node/.style={circle,draw,font=\sffamily\bfseries}, align=center]

  \baselineskip=0.1cm
  \lineskiplimit=4cm
  \lineskip=0.1cm

  \node[main node, initial, initial text={}] (a1) {$A_1$};
  \node[main node, shape = rectangle, right of = a1] (d1) {$D_1$};
  \node[below of = a1] (ai) {$\vdots$};
  \node[right of = ai] (di) {$\vdots$};
  \node[main node, initial, initial text={}, below of = ai] (an) {$A_n$};
  \node[main node, shape = rectangle, right of = an] (dn) {$D_n$};
	\draw[->,>=latex] (a1)->(d1) node[midway,above]{$\alpha$};
	\draw[->,>=latex] (an)->(dn) node[midway,above]{$\alpha$};
\end{tikzpicture}

\begin{myth}
La décision de la T-diagnosticabilité est PSPACE-dure.
\end{myth}
\end{frame}

\end{section}

\end{document}
