\documentclass[a4paper,10pt]{article}
\usepackage[utf8]{inputenc}
\usepackage[francais]{babel}
\usepackage{amsmath}
\usepackage{amsfonts}
\usepackage{amssymb}
\usepackage{tikz}
\usepackage{float}
\newtheorem{myth}{Th\'eor\`eme}
\newtheorem{mydef}{D\'efinition}
\title{Diagnosticabilit\'e avec co\^ut}

\begin{document}

\section*{Introduction}

Après avoir étudié les articles existants sur le sujet nous avons planifié deux axes de recherche pour étendre la notion de diagnosticabilité de deux manières différentes. Nous allons commencer par nous intéresser au diagnostic avec réparation puis nous essayerons de nous intéresser au diagnostic avec budget.

\section{Diagnostic avec r\'eparation}
\subsection{Rappel : Définition de la diagnosticabilité}
\begin{quote}
Soit un automate fini d\'eterministe $A$ appel\'e syst\`eme, un modèle de faute qui reconnait le langage de faute $L$ et une partition $\Sigma=\Sigma_o\sqcup \Sigma_{no}$. On dira qu'un \'ev\`enement est observable (resp. non-observable) s'il est dans $\Sigma_o$ (resp. $\Sigma_{no}$).

On dit que $L$ est diagnosticable dans le système $A$~\cite{SamSRST96}  si et seulement si $$\begin{array}{l}
\exists n \in \mathbb N, \forall w \in L,\\
\forall u \in \Sigma^*, \left(\left|\pi_{\Sigma_o}(u)\right|\ge n \land wu \in \mathcal L(A)\right) \implies\\
\forall v \in \mathcal L(A), \pi_{\Sigma_o}(v)=\pi_{\Sigma_o}(wu) \implies v \in L
\end{array}$$

Intuitivement, il existe une borne $n$ telle que pour toute ex\'ecution du système produisant un mot $w$ contenant une faute, si l'on attend $n$ nouveaux observables ($u$), alors tous les mots $v$ qui produisent la m\^eme observation que $wx$ contiennent une faute, ce qui veut dire que quand une faute a eu lieu, on peut toujours affirmer sans se tromper qu'elle eu lieu (au plus tard $n$ observables après qu'elle se soit produite).
\end{quote}

\subsection{Extension : Ajout des r\'eparations}

Le mod\`ele pr\'ec\'edent ne permet de mod\'eliser que les fautes permanentes. Nous envisageons donc de d\'efinir une nouvelle notion de diagnosticabilit\'e (que l'on appellerait $R$-diagnosticabilit\'e) qui prendrait en compte les r\'eparations. Un syst\`eme $R$-diagnosticable serait alors un syst\`eme pour lequel toutes les fautes sont d\'etect\'ees avant d'\^etre r\'epar\'ees et apr\`es un nombre born\'e d'\'ev\`enements observables. Nous chercherons \`a faire en sorte qu'un syst\`eme $R$-diagnosticable soit diagnosticable et qu'un syst\`eme diagnosticable dans lequel aucune r\'eparation ne peut avoir lieu le soit $R$-diagnosticable.

D'un point de vue technique, autoriser les r\'eparations revient \`a rel\^cher la contrainte de cl\^oture par extension ($L\Sigma^*\subseteq L$) impos\'ee au langage diagnostiqu\'e. En effet, si aucune r\'eparation n'est possible, une fois qu'une faut a eu lieu, elle ne peut pas dispara\^itre donc tous les mots partageant le m\^eme pr\'efixe sont fautifs mais lorsqu'une r\'eparation a lieu, le mot n'est pas fautif alors qu'un de ses pr\'efixes l'est puisqu'une faute a eu lieu.

Pour le moment, nous envisageons la d\'efinition suivante :

\begin{mydef}
  Une faute $f$ qui peut \^etre r\'epar\'ee par $r$ est $R$-diagnosticable sur un syst\`eme $A$ si et seulement si\\
  $\exists n \in \mathbb N,\exists Diag : \pi_o(\mathcal L(A)) \to 2^{\{tt, ff\}}\setminus \{\emptyset\}$,\\
  $\forall ufv\in \mathcal L(A)$ avec $v\in \left(\Sigma\setminus \{r\}\right)^*$,\\
  $\exists v'\le v, |\pi_o(v')| \le n \land Diag(\pi_o(ufv'))=\{tt\}$
\end{mydef}

Intuitivement, cela veut dire qu'il existe un diagnosticur $Diag$ qui prend une observation et dit si l'etat est sain ($\{tt\}$), fautif ($\{ff\}$) ou ambigu ($\{tt, ff\}$). On impose que ce diagnosticur soit capable, \'etant donn\'e une ex\'ecution ($ufvr$) contenant une faut et une r\'eparation de d\'etecter qu'une faute a effectivement eu lieu ($Diag(\pi_o(ufv'))=\{tt\}$) avant la r\'eparation (mais pas forc\'ement juste avant, n'importe quand entre la faute et la r\'eparation, d'o\`u l'utilisation de $v'$ qui est un pr\'efixe de $v$). L'existence du $n$ et la condition $|\pi_o(v')| \le n$ servent a forcer la d\'etection d'une faute non r\'epar\'ee en temps fini.

\subsection{R\'esultats envisag\'es}

Nous pensons pouvoir d\'ecider l'existence d'un tel couple $(n, Diag)$ en temps polynomial en utilisant une \emph{twin machine} $A\times A$ et envisageons $A\times det(A)$ comme diagnosticur.

%TODO ajouter pk AxA "appele twin machine". $cl_\varepsilon(A)\times det(A)$%

\section{Diagnostic avec budget}
\subsection{Motivation}

En pratique, l'observation d'un système a souvent un cout. Or la diagnosticabilité binaire habituelle ne permet pas de discriminer les systèmes que l'on peut diagnostiquer pour un coût très faible de ceux pour lesquels le diagnostic est plus couteux. Nous envisageons donc de raffiner la notion de diagnosticabilit\'e afin de prendre en compte son coût.


\subsection{D\'efinition}

Pour d\'efinir les nouvelles notions, on commence par introduire un $oracle$.

\begin{mydef}{Oracle}

  Soit $\{P,Q-P\}$ forme une partition de l'ensemble des \'etats du syst\`eme. Nous d\'efinissons un oracle comme \'etant une fonction $O_P: 2^Q \to \{True,False\} \times \mathbb{N}$, $O_P(A)$ qui prend un sous-ensembles $A$ de l'ensemble des \'etats dans l'automate et renvoie une valeur bool\'eenne vraie si et seulement si l'\'etat courant $q\in A$ est dans $A\cap P$ et une valeur $n \in \mathbb{N}$ indique le co\^ut de cette consultation. $A$ est ensemble des \'etats courants possible.

\end{mydef}

Si nous permettons la consultation de n'importe quel oracle, alors nous allons toujours pouvoir d\'eterminer si l'\'etat courant est fautif notamment avec l'oracle $O_F$ dont $F$ est l'ensemble des \'etats fautifs. Donc nous pouvons d\'efinir une notion de diagnosticabilit\'e avec oracles qui est moins fort que la diagnosticabilit\'e sans oracles. Voici un exemple qui montre que les diff\'erentes puissances des oracles existent pour le syst\`eme:

\begin{figure}[H]
  \begin{center}
    \begin{tikzpicture}
      \node[draw,circle](2) at (-2, 0){2};      
      \node[draw,circle](0) at (0, 0){0};
      \node[draw,rectangle](1) at (2, 0){1};
      \draw[->,>=latex] (0)->(2)node[midway,above]{$a$};
      \draw[->,>=latex] (0)->(1)node[midway,above]{$a$};
      \draw[->,>=latex] (0) edge[in=110,out=70,loop] node[above] {$a$} (0);
      \draw[->,>=latex] (1) edge[in=110,out=70,loop] node[above] {$a$} (1);
      \draw[->,>=latex] (2) edge[in=110,out=70,loop] node[above] {$a$} (2);
    \end{tikzpicture}
  \end{center}
\end{figure}

Dans l'exemple nous notons les \'etats fautifs par les rectangles et les \'etats non-fautifs par les cercles. Le syst\`eme initial n'est pas diagnosticable, et si nous pouvons utiliser l'oracle $O_{\{2\}}$, alors le système devient diagnosticable. Mais par contre si nous avons seulement le droit de consulter l'oracle moins puissant $O_{\{0,2\}}$ pour ce système, on voit bien que le système reste non-diagnosticable. Cet exemple montre qu'il existe différents niveaux de puissances des oracles pour les systèmes.

Ensuite nous allons d\'efinir un co\^ut moyen de la consultation pour un syst\`eme $k$-diagnosticable.

\begin{mydef}{Co\^ut moyen pour les syst\`emes $k$-diagnosticables}
  
  Nous posons la sommes des co\^uts \`a payer pour les consultations des oracles normalis\'ee par $k$ comme \'etant le co\^ut moyen des syst\`emes $k$-diagnosticables.
\end{mydef}

\subsection{Questions ouvertes}

Plusieurs questions restent à poser et plusieurs problèmes restent à résoudre :

\begin{itemize}
\item Nous pensons que toutes les consultations des oracles doivent avoir un co\^ut. Mais plusieurs questions se posent \`a ce niveau. Comment peut-on d\'efinir les diff\'erents co\^uts pour chaque consultation ? Une fois que nous aurons d\'efini les co\^uts de chaque consultation, comment peut-on trouver la strat\'egie pour laquelle le co\^ut du diagnostic soit minimal ?

\item Est-ce que nous devons encore ajouter des contraintes pour les consultations des oracles? Pour la m\^eme configuration, on ne peut consulter qu'une seule fois le m\^eme oracle par exemple.

\item Nous avons envie de d\'efinir un co\^ut moyen de la consultation de l'oracle. Mais dans des cas particulier cela va d\'efinir des syst\`emes qui sont en m\^eme temps non-diagnosticable et de co\^ut moyen nul. L'exemple ci-dessous est un tel exemple:

\begin{figure}[H]
  \begin{center}
    \begin{tikzpicture}
      \node[draw,circle](0) at (-1, 0){0};
      \node[draw,rectangle](1) at (1, 0){1};
      \draw[->,>=latex] (0)->(1)node[midway,above]{$a$};
      \draw[->,>=latex] (0) edge[in=110,out=70,loop] node[above] {$a$} (0);
      \draw[->,>=latex] (1) edge[in=110,out=70,loop] node[above] {$a$} (1);
    \end{tikzpicture}
  \end{center}
\end{figure}

  
\end{itemize}


\subsection{Les solutions envisag\'ees}

Nous allons tra\^iter les diff\'erents probl\`emes que nous avons pos\'es pr\'ec\'edemment

\subsubsection{Cas le plus simple}

Nous commen\c cerons par un oracle sur une partition de l'ensemble des états $P$ d\'efinie a priori et chaque consultation de l'oracle co\^ute $1$. La fonction oracle serait r\'eduite \`a $O_P : 2^Q \to \{True, False\}$. Notre but sera de trouver une strat\'egie telle que toutes les fautes sont diagnostiqu\'ees au plus apr\`es $k$ coups en utilisant le moins souvent possible l'oracle.

Nous pensons introduire une notion de $k$-diagnosticabilit\'e pour \'eviter les syst\`emes qui sont \`a la fois non-diagnosticables sans oracle et de co\^ut moyen nul.

\subsubsection{Cas avec plusieurs d'oracles}

Dans le cas plus g\'en\'eral, nous allons r\'esoudre le probl\`eme en donnant plusieurs oracles diff\'erents, et essayer de trouver la strat\'egie pour avoir un co\^ut moyen minimal pour diagnostiquer les syst\`emes $k$-diagnosticables.

\subsubsection{Extensions envisag\'ees}

Un système de diagnsotic avec oracle peut disposer d'un budget, qui décroit \`a chaque utilisation d'un oracle.

De la m\^eme mani\`ere, nous pouvons supposer que le budget peut \^etre recharg\'e par un \'ev\`enement particulier ou au bout d'un nombre de pas du syst\`eme.

\subsection{Plan de travail}

\begin{enumerate}
\item definir les oracles et les diff\'erentes puissances.
\item definir la notion de diagnosticabilité avec oracle.
\item definir la notion de diagnostiqueur avec oracle et son calcul ( on pense qu'on pourra utiliser la théorie des jeux).
\item g\'en\'eraliser les r\'esultats avec plusieurs oracles.
\end{enumerate}

\end{document}
