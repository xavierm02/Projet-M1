\documentclass[a4paper,10pt]{article}
\usepackage[utf8]{inputenc}
\usepackage[francais]{babel}
\usepackage{amsmath}
\usepackage{amsfonts}
\usepackage{amssymb}
\usepackage{tikz}
\usepackage{float}
\newtheorem{myth}{Th\'eor\`eme}
\newtheorem{mydef}{D\'efinition}
\title{Diagnosticabilit\'e avec co\^ut}

\begin{document}

\section*{Introduction}

Après avoir étudié les articles existants sur le sujet nous avons planifié deux axes de recherche pour étendre la notion de diagnosticabilité de deux manières différentes. Nous allons commencer par nous intéresser au diagnostic avec réparation puis nous essayerons de nous intéresser au diagnostic avec budget.

\section{Diagnostic avec r\'eparation}
\subsection{Rappel : Définition de la diagnosticabilité}
\begin{quote}
Soit un automate fini d\'eterministe $A$ appel\'e syst\`eme, un modèle de faute qui reconnait le langage de faute $L$ et une partition $\Sigma=\Sigma_o\sqcup \Sigma_{no}$. On dira qu'un \'ev\`enement est observable (resp. non-observable) s'il est dans $\Sigma_o$ (resp. $\Sigma_{no}$).

On dit que $L$ est diagnosticable dans le système $A$~\cite{SamSRST96}  si et seulement si $$\begin{array}{l}
\exists n \in \mathbb N, \forall w \in L,\\
\forall u \in \Sigma^*, \left(\left|\pi_{\Sigma_o}(u)\right|\ge n \land wu \in \mathcal L(A)\right) \implies\\
\forall v \in \mathcal L(A), \pi_{\Sigma_o}(v)=\pi_{\Sigma_o}(wu) \implies v \in L
\end{array}$$

Intuitivement, il existe une borne $n$ telle que pour toute ex\'ecution du système produisant un mot $w$ contenant une faute, si l'on attend $n$ nouveaux observables ($u$), alors tous les mots $v$ qui produisent la m\^eme observation que $wx$ contiennent une faute, ce qui veut dire que quand une faute a eu lieu, on peut toujours affirmer sans se tromper qu'elle eu lieu (au plus tard $n$ observables après qu'elle se soit produite).
\end{quote}

\subsection{Extension : Ajout des r\'eparations}

Le mod\`ele pr\'ec\'edent ne permet de mod\'eliser que les fautes permanentes, c'est-\`a-dire qu'une fois qu'une faut s'est produite, le syst\`eme reste fautif. Nous envisageons donc de d\'efinir une nouvelle notion de diagnosticabilit\'e (que l'on appellerait $R$-diagnosticabilit\'e) qui prendrait en compte les r\'eparations. Un syst\`eme $R$-diagnosticable serait alors un syst\`eme pour lequel toutes les fautes sont d\'etect\'ees avant d'\^etre r\'epar\'ees et apr\`es un nombre born\'e d'\'ev\`enements observables. Nous chercherons \`a faire en sorte qu'un syst\`eme $R$-diagnosticable soit diagnosticable et qu'un syst\`eme diagnosticable dans lequel aucune r\'eparation ne peut avoir lieu le soit $R$-diagnosticable.

D'un point de vue technique, autoriser les r\'eparations revient \`a rel\^acher la contrainte de cl\^oture par extension ($L\Sigma^*\subseteq L$) impos\'ee au langage diagnostiqu\'e. En effet, si aucune r\'eparation n'est possible, une fois qu'une faute a eu lieu, elle ne peut pas dispara\^itre donc tous les mots partageant le m\^eme pr\'efixe sont fautifs mais lorsqu'une r\'eparation a lieu, le mot n'est pas fautif alors qu'un de ses pr\'efixes l'est puisqu'une faute a eu lieu.

Pour le moment, nous envisageons la d\'efinition suivante :

\begin{mydef}
  Une faute $f$ qui peut \^etre r\'epar\'ee par $r$ est $R$-diagnosticable sur un syst\`eme $A$ si et seulement si\\
  $\exists n \in \mathbb N,\exists Diag : \pi_o(\mathcal L(A)) \to 2^{\{tt, ff\}}\setminus \{\emptyset\}$,\\
  $\forall ufv\in \mathcal L(A)$ avec $v\in \left(\Sigma\setminus \{r\}\right)^*$,\\
  $\left(ufvr \in L(A) \lor |\pi_o(v)| \ge n\right) \implies$\\
  $\exists v'\le v, Diag(\pi_o(ufv'))=\{tt\}$
\end{mydef}

Intuitivement, cela veut dire qu'il existe un diagnostiqueur $Diag$ qui prend une observation et dit si l'etat est sain ($\{tt\}$), fautif ($\{ff\}$) ou ambigu ($\{tt, ff\}$). On impose que ce diagnostiqueur soit capable, \'etant donn\'e une ex\'ecution $ufv$ contenant une faute $f$, de d\'etecter qu'une faute a effectivement eu lieu\\ ($Diag(\pi_o(ufv'))=\{tt\}$) avant la r\'eparation et en temps born\'e.

\subsection{R\'esultats envisag\'es}

La diagnosticabilit\'e peut \^etre d\'ecid\'ee en temps polynomial en d\'ecidant l'existence de cycle ambig\"u dans la twin machine $A\times A$. Nous pensons que la $R$-diagnosticabilit\'e est \'egalement d\'ecidable par un algorithme en temps polynomial utilisant la twin machine $A\times A$.

Nous allons essayer de le prouver et de construire un diagnosiqueur.

%Cependant, nous pensons que $\det(A)$ ne suffitra pas comme diagnostiqueur et envisageons donc d'utiliser $cl_\varepsilon(A)\times \det (A)$, le $cl_\varepsilon(A)$ servirait \`a 

%Nous pensons pouvoir d\'ecider l'existence d'un tel couple $(n, Diag)$ en temps polynomial en utilisant une \emph{twin machine} $A\times A$ et envisageons $A\times det(A)$ comme diagnosticur.

%TODO ajouter pk AxA "appele twin machine". $cl_\varepsilon(A)\times det(A)$%

\section{Diagnostic avec budget}
\subsection{Motivation}

En pratique, l'observation d'un système a souvent un cout. Or la diagnosticabilité binaire habituelle ne permet pas de discriminer les systèmes que l'on peut diagnostiquer pour un coût très faible de ceux pour lesquels le diagnostic est plus couteux. Nous envisageons donc de raffiner la notion de diagnosticabilit\'e afin de prendre en compte son coût.


\subsection{D\'efinition}

Pour d\'efinir les nouvelles notions, on commence par introduire un $oracle$.

\begin{mydef}{Oracle}

  Nous d\'efinissons un oracle comme \'etant une fonction $O: 2^Q\times Run \to \{True, False\} \times \mathbb{N}$, qui prend une ex\'ecution et une partition des \'etats courants possibles et renvoie une valeur bool\'eanne qui est vraie si et seulement si l'\'etat courant est dans la partition des \'etats donn\'es. La valeur $n \in \mathbb{N}$ indique le co\^ut de cette consultation.

\end{mydef}

Il apparaît clairement que si nous permettons la consultation de n'importe quel oracle infiniment souvent, alors nous allons toujours pouvoir d\'eterminer si l'\'etat courant est fautif. Nous pouvons donc consid\'erer cela comme une extension de la d\'efinition de la diagnosticabilit\'e. Il est possible trouver des syst\`emes qui ne sont pas diagnosticables, mais qui le deviennent lorsque l'observateur peut effectuer un unique appel à l'oracle. Comme dans l'exemple suivant:

\begin{figure}[H]
  \begin{center}
    \begin{tikzpicture}
      \node[draw,circle](0) at (-1, 0){0};
      \node[draw,rectangle](1) at (1, 0){1};
      \draw[->,>=latex] (0)->(1)node[midway,above]{$a$};
      \draw[->,>=latex] (0) edge[in=110,out=70,loop] node[above] {$a$} (0);
      \draw[->,>=latex] (1) edge[in=110,out=70,loop] node[above] {$a$} (1);
    \end{tikzpicture}
  \end{center}
\end{figure}

Dans l'exemple nous notons que les \'etats fautifs par les rectangles et les \'etat non-fautif par les cercles. Le syst\`eme initial n'est pas diagnosticabile, mais si nous pouvons d\'eambigu\"uer entre l'\'etat $1$ et l'\'etat $2$, alors le syst\`eme devient diagnosticable.

Ensuite nous allons d\'efinir un co\^ut moyen de la consultation pour un syst\`eme $k$-diagnosticable.

\begin{mydef}{Co\^ut moyen pour les syst\`emes $k$-diagnosticables}
  
  Nous posons la sommes des co\^uts \`a payer pour les consultations des oracles normalis\'ee par $k$ comme \'etant le co\^ut moyen des syst\`emes $k$-diagnosticables.
\end{mydef}

\subsection{Questions ouvertes}

Plusieurs questions restent à poser et plusieurs problèmes restent à résoudre :

\begin{itemize}
\item Nous pensons que toutes les consultations des oracles doivent avoir un co\^ut. Mais plusieurs questions se posent \`a ce niveau. Comment peut-on d\'efinir les diff\'erents co\^uts pour chaque consultation ? Une fois que nous aurons d\'efini les co\^uts de chaque consultation, comment peut-on trouver la strat\'egie pour laquelle le co\^ut du diagnostic soit minimal ?

\item Est-ce que nous devons encore ajouter des contraintes pour les consultations des oracles? Pour la m\^eme configuration, on ne peut consulter qu'une seule fois le m\^eme oracle par exemple.

\item Nous avons envie de d\'efinir un co\^ut moyen de la consultation de l'oracle. Mais dans des cas particulier cela va d\'efinir des syst\`emes qui sont en m\^eme temps non-diagnosticable et de co\^ut moyen nul. L'exemple ci-dessous est un tel exemple:

\begin{figure}[H]
  \begin{center}
    \begin{tikzpicture}
      \node[draw,circle](0) at (-1, 0){0};
      \node[draw,rectangle](1) at (1, 0){1};
      \draw[->,>=latex] (0)->(1)node[midway,above]{$a$};
      \draw[->,>=latex] (0) edge[in=110,out=70,loop] node[above] {$a$} (0);
      \draw[->,>=latex] (1) edge[in=110,out=70,loop] node[above] {$a$} (1);
    \end{tikzpicture}
  \end{center}
\end{figure}

  
\end{itemize}


\subsection{Plan de travail}

Nous allons tra\^iter les diff\'erents probl\`emes que nous avons pos\'es pr\'ec\'edemment

\subsubsection{Cas le plus simple}

Nous commen\c cerons par un oracle sur une partition de l'ensemble des états d\'efinie a priori. La fonction oracle serait r\'eduite \`a $O : Run \to \{True, False\}$. Nous consid\'ererons que chaque consultation de l'oracle co\^ute 1. Notre but sera de trouver une strat\'egie telle que toutes les fautes sont diagnostiqu\'ees au plus apr\`es $k$ coups en utilisant le moins souvent possible l'oracle.

Nous pensons introduire une notion de $k$-diagnosticabilit\'e pour \'eviter les syst\`emes qui sont \`a la fois non-diagnosticables et de co\^ut moyen nul.

\subsubsection{Cas avec les diff\'erents types d'oracles}

Dans le cas plus g\'en\'eral, nous allons r\'esoudre le probl\`eme en donnant plusieurs oracles diff\'erents, et essayer de trouver la strat\'egie pour avoir un co\^ut moyen minimal pour diagnostiquer les syst\`emes $k$-diagnosticables.

\subsubsection{Extensions envisag\'ees}
Les questions de budget rechargeable sont aussi envisag\'ees. Car il nous semble naturel qu'un budget puisse se recharger apr\`es un certain nombre de coups ou ap\`es avoir passé un \'etat sp\'ecifique.

\subsection{Quelques id\'ees et des pistes pour continuer}

Il nous semble tr\`es probable qu'il nous faudra utiliser des m\'ethodes de la th\'eorie des jeux pour trouver les strat\'egies de budget minimal.


\end{document}
