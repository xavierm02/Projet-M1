
%% bare_conf.tex
%% V1.3
%% 2007/01/11
%% by Michael Shell
%% See:
%% http://www.michaelshell.org/
%% for current contact information.
%%
%% This is a skeleton file demonstrating the use of IEEEtran.cls
%% (requires IEEEtran.cls version 1.7 or later) with an IEEE conference paper.
%%
%% Support sites:
%% http://www.michaelshell.org/tex/ieeetran/
%% http://www.ctan.org/tex-archive/macros/latex/contrib/IEEEtran/
%% and
%% http://www.ieee.org/

%%*************************************************************************
%% Legal Notice:
%% This code is offered as-is without any warranty either expressed or
%% implied; without even the implied warranty of MERCHANTABILITY or
%% FITNESS FOR A PARTICULAR PURPOSE! 
%% User assumes all risk.
%% In no event shall IEEE or any contributor to this code be liable for
%% any damages or losses, including, but not limited to, incidental,
%% consequential, or any other damages, resulting from the use or misuse
%% of any information contained here.
%%
%% All comments are the opinions of their respective authors and are not
%% necessarily endorsed by the IEEE.
%%
%% This work is distributed under the LaTeX Project Public License (LPPL)
%% ( http://www.latex-project.org/ ) version 1.3, and may be freely used,
%% distributed and modified. A copy of the LPPL, version 1.3, is included
%% in the base LaTeX documentation of all distributions of LaTeX released
%% 2003/12/01 or later.
%% Retain all contribution notices and credits.
%% ** Modified files should be clearly indicated as such, including  **
%% ** renaming them and changing author support contact information. **
%%
%% File list of work: IEEEtran.cls, IEEEtran_HOWTO.pdf, bare_adv.tex,
%%                    bare_conf.tex, bare_jrnl.tex, bare_jrnl_compsoc.tex
%%*************************************************************************

% *** Authors should verify (and, if needed, correct) their LaTeX system  ***
% *** with the testflow diagnostic prior to trusting their LaTeX platform ***
% *** with production work. IEEE's font choices can trigger bugs that do  ***
% *** not appear when using other class files.                            ***
% The testflow support page is at:
% http://www.michaelshell.org/tex/testflow/



% Note that the a4paper option is mainly intended so that authors in
% countries using A4 can easily print to A4 and see how their papers will
% look in print - the typesetting of the document will not typically be
% affected with changes in paper size (but the bottom and side margins will).
% Use the testflow package mentioned above to verify correct handling of
% both paper sizes by the user's LaTeX system.
%
% Also note that the "draftcls" or "draftclsnofoot", not "draft", option
% should be used if it is desired that the figures are to be displayed in
% draft mode.
%
\documentclass[conference]{IEEEtran}
% Add the compsoc option for Computer Society conferences.
%
% If IEEEtran.cls has not been installed into the LaTeX system files,
% manually specify the path to it like:
% \documentclass[conference]{../sty/IEEEtran}





% Some very useful LaTeX packages include:
% (uncomment the ones you want to load)


% *** MISC UTILITY PACKAGES ***
%
%\usepackage{ifpdf}
% Heiko Oberdiek's ifpdf.sty is very useful if you need conditional
% compilation based on whether the output is pdf or dvi.
% usage:
% \ifpdf
%   % pdf code
% \else
%   % dvi code
% \fi
% The latest version of ifpdf.sty can be obtained from:
% http://www.ctan.org/tex-archive/macros/latex/contrib/oberdiek/
% Also, note that IEEEtran.cls V1.7 and later provides a builtin
% \ifCLASSINFOpdf conditional that works the same way.
% When switching from latex to pdflatex and vice-versa, the compiler may
% have to be run twice to clear warning/error messages.






% *** CITATION PACKAGES ***
%
%\usepackage{cite}
% cite.sty was written by Donald Arseneau
% V1.6 and later of IEEEtran pre-defines the format of the cite.sty package
% \cite{} output to follow that of IEEE. Loading the cite package will
% result in citation numbers being automatically sorted and properly
% "compressed/ranged". e.g., [1], [9], [2], [7], [5], [6] without using
% cite.sty will become [1], [2], [5]--[7], [9] using cite.sty. cite.sty's
% \cite will automatically add leading space, if needed. Use cite.sty's
% noadjust option (cite.sty V3.8 and later) if you want to turn this off.
% cite.sty is already installed on most LaTeX systems. Be sure and use
% version 4.0 (2003-05-27) and later if using hyperref.sty. cite.sty does
% not currently provide for hyperlinked citations.
% The latest version can be obtained at:
% http://www.ctan.org/tex-archive/macros/latex/contrib/cite/
% The documentation is contained in the cite.sty file itself.






% *** GRAPHICS RELATED PACKAGES ***
%
\ifCLASSINFOpdf
  % \usepackage[pdftex]{graphicx}
  % declare the path(s) where your graphic files are
  % \graphicspath{{../pdf/}{../jpeg/}}
  % and their extensions so you won't have to specify these with
  % every instance of \includegraphics
  % \DeclareGraphicsExtensions{.pdf,.jpeg,.png}
\else
  % or other class option (dvipsone, dvipdf, if not using dvips). graphicx
  % will default to the driver specified in the system graphics.cfg if no
  % driver is specified.
  % \usepackage[dvips]{graphicx}
  % declare the path(s) where your graphic files are
  % \graphicspath{{../eps/}}
  % and their extensions so you won't have to specify these with
  % every instance of \includegraphics
  % \DeclareGraphicsExtensions{.eps}
\fi
% graphicx was written by David Carlisle and Sebastian Rahtz. It is
% required if you want graphics, photos, etc. graphicx.sty is already
% installed on most LaTeX systems. The latest version and documentation can
% be obtained at: 
% http://www.ctan.org/tex-archive/macros/latex/required/graphics/
% Another good source of documentation is "Using Imported Graphics in
% LaTeX2e" by Keith Reckdahl which can be found as epslatex.ps or
% epslatex.pdf at: http://www.ctan.org/tex-archive/info/
%
% latex, and pdflatex in dvi mode, support graphics in encapsulated
% postscript (.eps) format. pdflatex in pdf mode supports graphics
% in .pdf, .jpeg, .png and .mps (metapost) formats. Users should ensure
% that all non-photo figures use a vector format (.eps, .pdf, .mps) and
% not a bitmapped formats (.jpeg, .png). IEEE frowns on bitmapped formats
% which can result in "jaggedy"/blurry rendering of lines and letters as
% well as large increases in file sizes.
%
% You can find documentation about the pdfTeX application at:
% http://www.tug.org/applications/pdftex





% *** MATH PACKAGES ***
%
%\usepackage[cmex10]{amsmath}
% A popular package from the American Mathematical Society that provides
% many useful and powerful commands for dealing with mathematics. If using
% it, be sure to load this package with the cmex10 option to ensure that
% only type 1 fonts will utilized at all point sizes. Without this option,
% it is possible that some math symbols, particularly those within
% footnotes, will be rendered in bitmap form which will result in a
% document that can not be IEEE Xplore compliant!
%
% Also, note that the amsmath package sets \interdisplaylinepenalty to 10000
% thus preventing page breaks from occurring within multiline equations. Use:
%\interdisplaylinepenalty=2500
% after loading amsmath to restore such page breaks as IEEEtran.cls normally
% does. amsmath.sty is already installed on most LaTeX systems. The latest
% version and documentation can be obtained at:
% http://www.ctan.org/tex-archive/macros/latex/required/amslatex/math/





% *** SPECIALIZED LIST PACKAGES ***
%
%\usepackage{algorithmic}
% algorithmic.sty was written by Peter Williams and Rogerio Brito.
% This package provides an algorithmic environment fo describing algorithms.
% You can use the algorithmic environment in-text or within a figure
% environment to provide for a floating algorithm. Do NOT use the algorithm
% floating environment provided by algorithm.sty (by the same authors) or
% algorithm2e.sty (by Christophe Fiorio) as IEEE does not use dedicated
% algorithm float types and packages that provide these will not provide
% correct IEEE style captions. The latest version and documentation of
% algorithmic.sty can be obtained at:
% http://www.ctan.org/tex-archive/macros/latex/contrib/algorithms/
% There is also a support site at:
% http://algorithms.berlios.de/index.html
% Also of interest may be the (relatively newer and more customizable)
% algorithmicx.sty package by Szasz Janos:
% http://www.ctan.org/tex-archive/macros/latex/contrib/algorithmicx/




% *** ALIGNMENT PACKAGES ***
%
%\usepackage{array}
% Frank Mittelbach's and David Carlisle's array.sty patches and improves
% the standard LaTeX2e array and tabular environments to provide better
% appearance and additional user controls. As the default LaTeX2e table
% generation code is lacking to the point of almost being broken with
% respect to the quality of the end results, all users are strongly
% advised to use an enhanced (at the very least that provided by array.sty)
% set of table tools. array.sty is already installed on most systems. The
% latest version and documentation can be obtained at:
% http://www.ctan.org/tex-archive/macros/latex/required/tools/


%\usepackage{mdwmath}
%\usepackage{mdwtab}
% Also highly recommended is Mark Wooding's extremely powerful MDW tools,
% especially mdwmath.sty and mdwtab.sty which are used to format equations
% and tables, respectively. The MDWtools set is already installed on most
% LaTeX systems. The lastest version and documentation is available at:
% http://www.ctan.org/tex-archive/macros/latex/contrib/mdwtools/


% IEEEtran contains the IEEEeqnarray family of commands that can be used to
% generate multiline equations as well as matrices, tables, etc., of high
% quality.


%\usepackage{eqparbox}
% Also of notable interest is Scott Pakin's eqparbox package for creating
% (automatically sized) equal width boxes - aka "natural width parboxes".
% Available at:
% http://www.ctan.org/tex-archive/macros/latex/contrib/eqparbox/





% *** SUBFIGURE PACKAGES ***
%\usepackage[tight,footnotesize]{subfigure}
% subfigure.sty was written by Steven Douglas Cochran. This package makes it
% easy to put subfigures in your figures. e.g., "Figure 1a and 1b". For IEEE
% work, it is a good idea to load it with the tight package option to reduce
% the amount of white space around the subfigures. subfigure.sty is already
% installed on most LaTeX systems. The latest version and documentation can
% be obtained at:
% http://www.ctan.org/tex-archive/obsolete/macros/latex/contrib/subfigure/
% subfigure.sty has been superceeded by subfig.sty.



%\usepackage[caption=false]{caption}
%\usepackage[font=footnotesize]{subfig}
% subfig.sty, also written by Steven Douglas Cochran, is the modern
% replacement for subfigure.sty. However, subfig.sty requires and
% automatically loads Axel Sommerfeldt's caption.sty which will override
% IEEEtran.cls handling of captions and this will result in nonIEEE style
% figure/table captions. To prevent this problem, be sure and preload
% caption.sty with its "caption=false" package option. This is will preserve
% IEEEtran.cls handing of captions. Version 1.3 (2005/06/28) and later 
% (recommended due to many improvements over 1.2) of subfig.sty supports
% the caption=false option directly:
%\usepackage[caption=false,font=footnotesize]{subfig}
%
% The latest version and documentation can be obtained at:
% http://www.ctan.org/tex-archive/macros/latex/contrib/subfig/
% The latest version and documentation of caption.sty can be obtained at:
% http://www.ctan.org/tex-archive/macros/latex/contrib/caption/




% *** FLOAT PACKAGES ***
%
%\usepackage{fixltx2e}
% fixltx2e, the successor to the earlier fix2col.sty, was written by
% Frank Mittelbach and David Carlisle. This package corrects a few problems
% in the LaTeX2e kernel, the most notable of which is that in current
% LaTeX2e releases, the ordering of single and double column floats is not
% guaranteed to be preserved. Thus, an unpatched LaTeX2e can allow a
% single column figure to be placed prior to an earlier double column
% figure. The latest version and documentation can be found at:
% http://www.ctan.org/tex-archive/macros/latex/base/



%\usepackage{stfloats}
% stfloats.sty was written by Sigitas Tolusis. This package gives LaTeX2e
% the ability to do double column floats at the bottom of the page as well
% as the top. (e.g., "\begin{figure*}[!b]" is not normally possible in
% LaTeX2e). It also provides a command:
%\fnbelowfloat
% to enable the placement of footnotes below bottom floats (the standard
% LaTeX2e kernel puts them above bottom floats). This is an invasive package
% which rewrites many portions of the LaTeX2e float routines. It may not work
% with other packages that modify the LaTeX2e float routines. The latest
% version and documentation can be obtained at:
% http://www.ctan.org/tex-archive/macros/latex/contrib/sttools/
% Documentation is contained in the stfloats.sty comments as well as in the
% presfull.pdf file. Do not use the stfloats baselinefloat ability as IEEE
% does not allow \baselineskip to stretch. Authors submitting work to the
% IEEE should note that IEEE rarely uses double column equations and
% that authors should try to avoid such use. Do not be tempted to use the
% cuted.sty or midfloat.sty packages (also by Sigitas Tolusis) as IEEE does
% not format its papers in such ways.





% *** PDF, URL AND HYPERLINK PACKAGES ***
%
%\usepackage{url}
% url.sty was written by Donald Arseneau. It provides better support for
% handling and breaking URLs. url.sty is already installed on most LaTeX
% systems. The latest version can be obtained at:
% http://www.ctan.org/tex-archive/macros/latex/contrib/misc/
% Read the url.sty source comments for usage information. Basically,
% \url{my_url_here}.





% *** Do not adjust lengths that control margins, column widths, etc. ***
% *** Do not use packages that alter fonts (such as pslatex).         ***
% There should be no need to do such things with IEEEtran.cls V1.6 and later.
% (Unless specifically asked to do so by the journal or conference you plan
% to submit to, of course. )


% correct bad hyphenation here
\hyphenation{op-tical net-works semi-conduc-tor}

\usepackage[utf8]{inputenc}
\usepackage[francais]{babel}
\usepackage{amsmath}
\usepackage{amsfonts}
\usepackage{amssymb}
\usepackage{tikz}
\usepackage{wasysym}
\usepackage{mathtools}
\usepackage{stmaryrd}
\usepackage{tikz}
\usepackage{float}
\newtheorem{mydef}{D\'efinition}
\newtheorem{myth}{Th\'eor\`eme}
\newtheorem{myco}{Corollaire}
\newtheorem{myconj}{Conjecture}
\newtheorem{mydis}{Discussion}

\usepackage{amssymb}\usepackage{varwidth}
\usetikzlibrary{arrows,positioning,automata,shadows}

\newcommand{\es}{\square}
\newcommand{\fs}{\blacksquare}
\newcommand{\ec}{\ocircle}
\newcommand{\enum}[2]{\llbracket #1, #2 \rrbracket}

\begin{document}
%
% paper title
% can use linebreaks \\ within to get better formatting as desired
\title{T-diagnosticabilit\'e}


% author names and affiliations
% use a multiple column layout for up to three different
% affiliations
\author{Fran\c{c}ois \textsc{Godi}, Chen \textsc{Qian} et Xavier \textsc{Montillet}}

% conference papers do not typically use \thanks and this command
% is locked out in conference mode. If really needed, such as for
% the acknowledgment of grants, issue a \IEEEoverridecommandlockouts
% after \documentclass

% for over three affiliations, or if they all won't fit within the width
% of the page, use this alternative format:
% 
%\author{\IEEEauthorblockN{Michael Shell\IEEEauthorrefmark{1},
%Homer Simpson\IEEEauthorrefmark{2},
%James Kirk\IEEEauthorrefmark{3}, 
%Montgomery Scott\IEEEauthorrefmark{3} and
%Eldon Tyrell\IEEEauthorrefmark{4}}
%\IEEEauthorblockA{\IEEEauthorrefmark{1}School of Electrical and Computer Engineering\\
%Georgia Institute of Technology,
%Atlanta, Georgia 30332--0250\\ Email: see http://www.michaelshell.org/contact.html}
%\IEEEauthorblockA{\IEEEauthorrefmark{2}Twentieth Century Fox, Springfield, USA\\
%Email: homer@thesimpsons.com}
%\IEEEauthorblockA{\IEEEauthorrefmark{3}Starfleet Academy, San Francisco, California 96678-2391\\
%Telephone: (800) 555--1212, Fax: (888) 555--1212}
%\IEEEauthorblockA{\IEEEauthorrefmark{4}Tyrell Inc., 123 Replicant Street, Los Angeles, California 90210--4321}}




% use for special paper notices
%\IEEEspecialpapernotice{(Invited Paper)}




% make the title area
\maketitle


\begin{abstract}
%\boldmath
On considère des systèmes à évènement discret qui peuvent commettre des fautes. La diagnosticabilité est une propri\'et\'e qui exprime la capacit\'e \`a d\'etecter les fautes du syst\`eme \`a partir d'une connaissance partielle de son \'etat. La notion classique de diagnosticabilit\'e ne permet de mod\'eliser que des fautes permanentes, et le langage des fautes est donc suppos\'e extension-clos. Dans ce papier, nous rel\^achons cette hypoth\`ese afin de permettre la mod\'elisation de fautes r\'eparables. Pour ce faire, nous introduisons la T-diagnosticabilit\'e qui formalise le fait que toute faute est d\'etect\'ee en temps born\'e et avant d'\^etre r\'epar\'ee.

Nous montrons que la T-diagnosticabilit\'e \'etend bien la notion de diagnosticabilit\'e, c'est-\`a-dire que pour un syst\`eme dont les fautes sont permanentes, les notions de T-diagnosticabilit\'e et de diagnosticabilit\'e coincident.

Nous proposons un algorithme de d\'ecision PSPACE pour la T-diagnosticabilit\'e et une construction de T-diagnostiqueur.
\end{abstract}
% IEEEtran.cls defaults to using nonbold math in the Abstract.
% This preserves the distinction between vectors and scalars. However,
% if the conference you are submitting to favors bold math in the abstract,
% then you can use LaTeX's standard command \boldmath at the very start
% of the abstract to achieve this. Many IEEE journals/conferences frown on
% math in the abstract anyway.

% no keywords




% For peer review papers, you can put extra information on the cover
% page as needed:
% \ifCLASSOPTIONpeerreview
% \begin{center} \bfseries EDICS Category: 3-BBND \end{center}
% \fi
%
% For peerreview papers, this IEEEtran command inserts a page break and
% creates the second title. It will be ignored for other modes.
\IEEEpeerreviewmaketitle



\section*{Introduction}
% no \IEEEPARstart
De nos jours, les syst\`emes d'information ont tendance \`a \^etre trop complexes pour que l'on puisse se convaincre de leur sûret\'e. On les mod\'elise donc formellement pour prouver des propriétés. Notamment, on a souvent besoin de détecter des problèmes en temps réel dans des systèmes à évènements discrets. Mais détecter une partie des fautes en temps réel n'est généralement pas suffisant, on veut donc pouvoir vérifier a priori que l'on va être capable de détecter toutes les fautes du système.  La diagnosticabilit\'e est une propri\'et\'e qui exprime la capacit\'e \`a d\'etecter toutes les fautes du syst\`eme \`a partir d'une connaissance partielle de son \'etat.

La notion classique de diagnosticabilit\'e\cite{SamSRST96} ne s'applique qu'aux systèmes où les fautes sont définitives. Dans ce papier, nous généralisons cette notion afin de prendre en compte les systèmes où les fautes peuvent être r\'eparées. Notre nouvelle notion, la T-diagnosticabilit\'e, formalise le fait que toute faute est détectée en temps born\'e et avant d'avoir été r\'epar\'ee.

Voici comment s'organise ce rapport : Dans la section I, on rappelle les prérequis, sur les automates puis la notion classique de diagnosticabilité. Dans la section II, on introduit la définition de la T-diagnosticabilit\'e, de T-diagnostiqueur et on montre que la T-diagnosticabilit\'e généralise bien la diagnosticabilité. Dans la section III, nous montrons que la diagnosticabilité est $PSPACE$, nous expliquons pourquoi nous pensons qu'elle est $PSPACE-Complet$ et nous donnerons un algorithme $PSPACE$ pour décider la diagnosticabilité.

\section{Pr\'erequis}

\subsection{Automates}

\begin{mydef}[Automate]
Un automate est un 5-uplet $A = (Q, \Sigma, \Delta, q_0, F)$ o\`u
\begin{itemize}
	\item $Q$ est l'ensemble des états du système.
	\item $\Sigma$ est l'ensemble des événements.
	\item $\Delta \subseteq (Q \times (\Sigma \cup \{\varepsilon\}) \times Q)$ est une relation de transition (o\`u $\varepsilon$ est le mot vide, c'est a dire que pour tout mot $u\in\Sigma^*$, $\varepsilon u = u = u\varepsilon$).
	\item $q_0 \in Q$ l'état initial.
	\item $F \subseteq Q$ l'ensemble des \'etats finaux.
\end{itemize}
\end{mydef}

Un automate fini est un automate tel que $Q$ est \emph{fini}.
\paragraph{}
\'Etant donn\'es $p,q \in Q$ et $l\in \Sigma \cup \{\varepsilon\}$, on notera $p \underset{\Delta}{\overset{l}{{\to}}}q$ pour $(p,l,q) \in \Delta$. On notera \'egalement $q_0 \underset{\Delta}{\overset{l_0}{{\to}}} \dots \underset{\Delta}{\overset{l_n}{{\to}}} q_{n+1}$ pour $\forall k \in \enum{0}{n}, q_k 
\underset{\Delta}{\overset{l_k}{{\to}}} q_{k+1}$. Pour simplifier, le $\Delta$ sera omis lorsqu'il est \'evident.
\paragraph{}
On dit qu'un automate est \emph{sans $\varepsilon$-transitions} si $\Delta \subseteq (Q\times \Sigma \times Q)$.

Un automate \emph{complet} est un automate tel que pour tout \'etat $q$ et pour toute lettre $l$ il existe une transition partant de $q$ \'etiquet\'ee par $l$. Formellement : $\forall q \in Q, \forall l \in \Sigma, \exists p\in Q, q\overset{l}{\to}p$

Un automate \emph{d\'eterministe} est un automate sans $\varepsilon$-transitions tel que $\Delta$ peut \^etre vue comme une fonction de $Q\times \Sigma$ dans $Q$: $$\forall p,q,q' \in Q, \forall l \in \Sigma, p\overset{l}{{\to}}q \land p\overset{l}{{\to}}q' \implies q=q'$$

\begin{mydef}[Langage reconnu par un automate]
Un mot $l_1...l_n \in \left(\Sigma\cup \{\varepsilon\}\right)^*$ est reconnu par un automate $A$ si il existe $n \in \mathbb N$ et $q_1,...,q_n \in Q$ tels que $q_0 \overset{l_1}{{\to}} q_1 \overset{l_2}{{\to}} \dots \overset{l_n}{{\to}} q_{n}$ et $q_{n} \in F$.

Le langage $\mathcal{L}(A)$ reconnu par un automate $A$ est l'ensemble des mots de $\Sigma^*$ reconnus par $A$.
\end{mydef}

\begin{figure}[H]
  \caption{exemple d'un automate}
  \begin{center}
    \begin{tikzpicture}
      \node[] (0) at (-1,0) {};
      \node[draw,circle] (1) at (0,0) {O};
      \node[draw,circle] (2) at (1.5,1) {N};
      \node[draw,circle] (3) at (3,0) {E};
      \node[draw,circle] (4) at (1.5,-1) {S};
      \draw[->,>=latex] (0)->(1) {}; 
      \draw[->,>=latex] (1)->(2) node[midway,above]{$a$};
      \draw[->,>=latex] (2)->(3) node[midway,above]{$a,b$};
      \draw[->,>=latex] (3)->(4) node[midway,above]{$b$};
      \draw[->,>=latex] (4)->(1) node[midway,above]{$a,b$};
      \draw [->,>=latex] (1) edge[in=100,out=80,loop] node[above] {$a$} (1);
      \draw [->,>=latex] (3) edge[in=100,out=80,loop] node[above] {$b$} (3);
    \end{tikzpicture}
  \end{center}
\end{figure}

\begin{mydef}[$\varepsilon$-clôture]
L'$\varepsilon$-clôture \`a gauche  d'un automate $A$ est l'automate $\operatorname{cl}_\varepsilon(A) = (Q, \Sigma, \Delta', q_0, F)$ et l'on a $p \underset{\Delta'}{\overset{l}{{\to}}} r$ si et seulement si il existe $n\ge 1$ et $q_1,\dots,q_n \in Q,$ tels que $q_1 = p$, $q_n = r$ et $q_1 \underset{\Delta}{\overset{\varepsilon}{{\to}}} \dots \underset{\Delta}{\overset{l}{{\to}}} q_n$.
\end{mydef}
On peut prouver que $\mathcal L(\operatorname{cl}_\varepsilon(A))=\mathcal L(A)$.

\begin{mydef}[Compl\'et\'e d'un automate]
Le compl\'et\'e d'un automate déterminisé complet $A = (Q, \Sigma, \Delta, q_0, F)$ sans $\varepsilon$-transitions est l'automate $\operatorname{complete}(A) = (Q\cup\{\bot\}, \Sigma, \Delta', q_0, F)$ o\`u $\bot\not\in Q$ est un nouvel \'etat et $\Delta'$ est d\'efinie par :
$$\forall l \in \Sigma,\forall p,q \in Q, \left(p\underset{\Delta'}{\overset{l}{{\to}}}q \iff p\underset{\Delta}{\overset{l}{{\to}}}q\right)$$
$$\forall l \in \Sigma, \forall p \in Q, \left(p\underset{\Delta'}{\overset{l}{{\to}}}\bot \iff \left(\forall q \in Q, p\underset{\Delta}{\overset{l}{{\not\to}}}q\right)\right)$$

$$\forall l \in \Sigma, \forall q \in Q,\bot\underset{\Delta'}{\overset{l}{{\not\to}}}q$$

$$\forall l \in \Sigma, \bot\underset{\Delta'}{\overset{l}{{\to}}}\bot$$
\end{mydef}

On peut prouver que $\mathcal L(\operatorname{complete}(A)) = \mathcal L(A)$ et que $\operatorname{complete}(A)$ est complet.

\begin{mydef}[D\'eterminis\'e d'un automate]
Le d\'eterminis\'e d'un automate $A = (Q, \Sigma, \Delta, q_0, F)$ sans $\varepsilon$-transitions est l'automate $$\det(A) = (\mathcal P(Q), \Sigma, \Delta', \{q_0\}, \mathcal P(F))$$ o\`u $\mathcal P(X)=\{Y\mid Y \subseteq X\}$ est l'ensemble des parties de $X$ et $\Delta'$ est d\'efinie par :
$$\forall l \in \Sigma,\forall p',q' \in \mathcal P(Q), \left(p'\underset{\Delta'}{\overset{l}{{\to}}}q' \iff \left( \exists p \in p', \exists q \in q', p\underset{\Delta}{\overset{l}{{\to}}}q\right)\right)$$
\end{mydef}

On peut prouver que $\mathcal L(\det(A))=\mathcal L(A)$ et que $\det(A)$ est un automate d\'eterministe.

\begin{mydef}[Compl\'ementaire d'un automate]
Le compl\'ementaire d'un automate $A = (Q, \Sigma, \Delta, q_0, F)$ sans $\varepsilon$-transitions, d\'eterministe et complet est l'automate $$\overline{A} = (Q, \Sigma, \Delta, q_0, Q\setminus F)$$
\end{mydef}

On peut prouver que $\mathcal L\left(\overline{A}\right) = \Sigma^* \setminus \mathcal L(A)$.

\begin{mydef}[Produit synchrone]
Soient $A = (Q, \Sigma, \Delta, q_0, F)$ et $A' = (Q', \Sigma', \Delta', q_0', F')$ deux automates sans $\varepsilon$. On définit le produit synchrone de ces automates comme \'etant $$A\times A' = (Q \times Q', \Sigma \cap \Sigma', \Delta'', (q_0, q_0'), F\times F')$$ o\`u $\Delta''$ est d\'efinie par $(p,p')\overset{l}{{\to}}(q,q')$ pour $l\in \Sigma\cap \Sigma'$ si et seulement si $p\overset{l}{{\to}}q$ et $p'\overset{l}{{\to}}q'$.
\end{mydef}
On peut prouver que $\mathcal L(A\times B) = \mathcal L(A) \cap \mathcal L(B)$.

\begin{mydef}[Langage extension-clos]
Un langage $L$ est dit extension-clos si $\forall u \in L, \forall v \in \Sigma^*, uv \in L$.
\end{mydef}

\begin{mydef}[Projection]
Soit $\Sigma'\subseteq \Sigma$. On d\'efinit la projection $\pi_{\Sigma'}$ de $\Sigma^*$ dans $\Sigma'^*$ par:

$$\begin{array}{llll}
&\pi_{\Sigma'}(\varepsilon) &= &\varepsilon\\
\forall a \in \Sigma, \forall u \in \Sigma^*, &\pi_{\Sigma'}(au) &= &\left\{\begin{array}{ll}
a\pi_{\Sigma'}(u) &\text{ si } a \in \Sigma'\\
\pi_{\Sigma'}(u) &\text{ si } a \not\in \Sigma'
\end{array}\right.
\end{array}$$
\end{mydef}

\subsection{Diagnosticabilité classique}
\'Etant donné un système qui peut commettre des fautes, il est important de s'assurer que ce qu'on observe du système suffit pour détecter une faute, c'est à dire de s'assurer que le système est \emph{diagnosticable}. On appelle \emph{diagnosticabilité} le problème qui consiste à dire si, oui ou non, les évènements observables d'un système suffisent à détecter les erreurs du système. La diagnosticabilité introduite par Sampath et autres\cite{SamSRST96} est décidable polynomialement mais est réservée aux systèmes dont les fautes sont définitives.
\newline
Soit un automate fini d\'eterministe $S$ dont tous les états sont finaux appel\'e syst\`eme. Les états de $S$ sont marqués fautifs ou non fautifs et, comme les fautes ne sont pas réparables, les successeurs d'un état fautif sont toujours fautifs. On note  $L_F(S)$, le langage de faute, l'ensemble des mots que le système peut reconnaitre en terminant sur un état fautif. Les lettres de l'alphabet $\Sigma=\Sigma_o\sqcup \Sigma_{no}$ de $S$ représentent les évènements. On dira qu'un \'ev\`enement est observable (resp. non-observable) s'il est dans $\Sigma_o$ (resp. $\Sigma_{no}$).

On dit que $S$ est diagnosticable si et seulement si

$$\begin{array}{l}
\exists n_0 \in \mathbb N, \forall x \in L_F(S),\\
\forall y \in \Sigma^*, \left(||y||_{\Sigma_o} \ge n \land xy \in  L(S)\right) \implies\\
\forall z \in L(S), \pi_{\Sigma_o}(z)=\pi_{\Sigma_o}(xy) \implies z \in L_F(S)
\end{array}$$

Intuitivement, il existe une borne $n_0$ telle que pour toute ex\'ecution du système produisant un mot $x$ contenant une faute, si l'on attend $n_0$ nouveaux observables ($y$), alors tous les mots $z$ qui produisent la m\^eme observation que $xy$ contiennent une faute, ce qui veut dire que quand une faute a eu lieu, on peut toujours affirmer sans se tromper qu'elle eu lieu (au plus tard $n_0$ observables après qu'elle se soit produite).

%R\'eduction : Langage de fautes $\Rightarrow$ \'etats fautifs / non-fautifs

%R\'eduction aux mots fautifs minimaux

\section{T-Diagnosticabilité}

On va introduire une nouvelle notion de diagnosticabilité, la T-diagnosticabilité,  qui généralise la diagnosticabilité en autorisant les fautes à se réparer. Dans cette section on définit la T-diagnosticabilité et on montre qu'elle généralise bien la diagnosticabilité.

\subsection{T-diagnosticabilité, Transient fault diagnosticability}
On part toujours de $S$, un système dont les évènements sont dans $\Sigma = \Sigma_o \ \bigsqcup \Sigma_{no}$. Mais on peut maintenant avoir une transition d'un état fautif vers un état non fautif, ainsi $L_F(S)$ n'est plus extension clos.

\begin{mydef}[Langage de faute minimal]
$$\begin{array}{lll}
L_{F}^{MIN}(S)&=\{u \in L_F(S) \mid & \forall v,w \in \Sigma^*, u = vw \\
&&\land v.\text{Pref}(w)\subseteq L_F(S) \implies w = \varepsilon \}\\
 & = \{u \in L_F(S) \mid \\
 &&\forall a \in \Sigma, \forall v \in \Sigma^*, u = va \implies\\
 && v \not \in L_F^{MIN}(S)\}
\end{array}$$
\end{mydef}

Le langage de faute minimale permet de considérer qu'une faute se produit seulement au moment exact où le système commence à être en faute, on ne compte pas une faute à nouveau si le système reste en faute.

\begin{mydef}[T-diagnosticabilité, par Hervé Marchand]

$$\begin{array}{l}
\forall u \in L^{MIN}_F(S),\ \exists n \in \mathbb{N}, \forall v \in \Sigma^*.\\
(uv \in L(S) \ \wedge \  ||v||_{\Sigma_o}\geq n) \Rightarrow \ \exists v' \leq v. \\
(u.Pref(v') \subseteq L_F(S) \ \wedge \  [uv'] \cap L(S) \subseteq L_F(S))
\end{array}$$
\end{mydef}

Intuitivement, pour chaque faute $u$, il existe une borne $n$ telle que si l'on a attendu $n$ nouveaux observables ($v$), alors il y a eu un moment où l'on a observé $uv'$ et où tous les mots qui pouvaient produire la m\^eme observation que $uv'$ contiennent une faute. Ainsi, lorsque une faute a eu lieu, on peut toujours affirmer sans se tromper qu'elle eu lieu (au plus tard $n$ observables après qu'elle se soit produite).

Prenons un exemple de T-diagnosticabilit\'e :

\begin{figure}[H]
  \caption{V\'erification de T-diagnosticabilit\'e. Les \'etats carr\'ees repr\'esente les \'etats fautifs et les circles repr\'esente les \'etats non-fautifs}
  \begin{center}
    \begin{tikzpicture}
      \node[] (0) at (-1,0) {};
      \node[draw,circle] (1) at (0,0) {O};
      \node[draw,rectangle] (2) at (1.5,1) {N};
      \node[draw,rectangle] (3) at (3,0) {E};
      \node[draw,circle] (4) at (1.5,-1) {S};
      \draw[->,>=latex] (0)->(1) {}; 
      \draw[->,>=latex] (1)->(2) node[midway,above]{$a$};
      \draw[->,>=latex] (2)->(3) node[midway,above]{$a,b$};
      \draw[->,>=latex] (3)->(4) node[midway,below]{$b$};
      \draw[->,>=latex] (4)->(1) node[midway,below]{$a,b$};
      \draw [->,>=latex] (1) edge[in=100,out=80,loop] node[above] {$a$} (1);
      \draw [->,>=latex] (3) edge[in=100,out=80,loop] node[above] {$b$} (3);
      \draw [->,>=latex,dashed] (1) edge [in=110,out=0] node[midway,above]{$b$} (4) ;
    \end{tikzpicture}
  \end{center}
\end{figure}

Ce syst\`eme sans les deux fl\`ches $c$ est T-diagnosticable. Si on ajoute les deux fl\`eche $c$. Consid\'erons les deux ex\'ecutions suivantes:

\begin{figure}[H]
  \caption{Deux ex\'ecutions qui montrent que le syst\`eme est non-T-diagnosticable}
  \begin{center}
    \begin{tikzpicture}
      \node[](0) at (-1,0) {};
      \node[draw,circle] (1) at (0,0) {O};
      \node[draw,rectangle] (2) at (2,0) {N};
      \node[draw,rectangle] (3) at (4,0) {E};
      \node[draw,circle] (4) at (6,0) {S};
      \draw[->,>=latex] (0)->(1) {};
      \draw[->,>=latex] (1)->(2) node[midway,above]{$a$};
      \draw[->,>=latex] (2)->(3) node[midway,above]{$a$};
      \draw[->,>=latex] (3)->(4) node[midway,above]{$b$};

      
      \node[](5) at (-1,-2) {};
      \node[draw,circle] (6) at (0,-2) {O};
      \node[draw,circle] (7) at (2,-2) {O};
      \node[draw,circle] (8) at (4,-2) {O};
      \node[draw,circle] (9) at (6,-2) {S};
      
      \draw[->,>=latex] (5)->(6) {};
      \draw[->,>=latex] (6)->(7) node[midway,above]{$a$};
      \draw[->,>=latex] (7)->(8) node[midway,above]{$a$};
      \draw[->,>=latex] (8)->(9) node[midway,above]{$b$};
    \end{tikzpicture}
  \end{center}
\end{figure}

Nous voyons que dans ces deux ex\'ecutions, la faute qui est pass\'ee par $N,E$ dans la premi\`ere ex\'ecution n'est jamais d\'etect\'ee.

\subsection{T-Diagnostiqueur}
Comme pour la diagnosticabilité classique, pour détecter effectivement les fautes sur des séquences d'évènements, on utilise un diagnostiqueur.
La construction du T-diagnostiqueur consiste seulement à d\'eterminiser l'automate $S$. Pour utiliser ce T-diagnostiqueur sur une séquence d'évènements, on lit le mot correspondant avec le déterminisé et quand on arrive dans un \'etat qui correspond uniquement à des \'etats $\Box$, on peut lever une alarme pour indiquer qu'il y a eu une faute. Si le système est T-diagnosticable alors toute faute sera suivie d'un instant non-ambigu où tous les états dans lesquels pourraient être le système sont fautifs. LE T-diagnostiqueur ne manque donc aucune faute lorsque le système est T-diagnosticable.

\subsection{La T-diagnosticabilité généralise la diagnosticabilité}

\begin{myth}[Diag $\Rightarrow$ T-diag]

Soit $S$, un système diagnosticable, montrons que $S$ est T-diagnosticable. 
\paragraph{}
\noindent $\forall u \in L_{F}(S)$, on prend $n := n_0$, $\forall v \in \Sigma^*$, on prend $v' := v$. 
\begin{equation}
v'\leq v \ \wedge \  u.Pref(v') \subseteq L_F(S)
\end{equation}
On sait que (1) est vrai puisque $v'=v$, que $L_F(S)$ est suffixe clos et que $u \in L_F(S)$.
\begin{equation}
(uv \in L(S) \ \wedge \  ||v||_{\Sigma_o}\geq n)
\end{equation}
De plus, comme $S$ est diag, si l'on a (2) alors tout mot de $L(S)$ observationellement équivalent à $uv$ est fautif. Donc l'ensemble de ces mots est inclus dans $L_F(S)$ et l'on a :
\begin{equation}
[uv'] \cap L(S) \subseteq L_F(S)
\end{equation}
Ainsi, on a $(1) \wedge [ (2)  \Rightarrow (3) ]$ donc $(2)  \Rightarrow [ (1) \wedge (3) ]$, on retrouve exactement la définition de la T-diagnosticabilité, $S$ est T-diagnosticable.
\end{myth}
\begin{myth}[T-diag sans r\'eparation $\Rightarrow$ diag]
Soit $S$, un système T-diag sans réparation (sans transition d'un état fautif vers un état non fautif), montrons que $S$ est diagnosticable.
\paragraph{}
\noindent On prend $n_0:= \max\limits_{u} n$ et
$\forall x \in L_F(S),\ y \in \Sigma^*,\ z \in L(S),$ si l'on a :
\begin{equation}
||y||_{\Sigma_o} \ge n \land xy \in  L(S) 
\end{equation}
Alors l'ensemble des mots observationnellement équivalents à $xy$ est inclus dans $L_F(S)$ car $S$ est T-diag. Donc on a :
\begin{equation}
 \pi_{\Sigma_o}(z)=\pi_{\Sigma_o}(xy) \implies z \in L_F(S)
\end{equation}
Ainsi on a $(4) \Rightarrow (5)$, on retrouve la définition de la diagnosticabilité, $S$ est diagnosticable.
\end{myth}
\section{Décision de la T-diagnosticabilité}
\begin{figure}
\caption{Description des paires d'excécutions compatibles avec la T-diagnosticabilité}
\label{a2diag}
\begin{tikzpicture}[->,>=stealth',shorten >=1pt,auto,node distance=3cm,
  thick,main node/.style={circle,draw,font=\sffamily\Large\bfseries}, align=center]

  \baselineskip=0.1cm
  \lineskiplimit=4cm
  \lineskip=0.1cm

  \node[main node] (4) [fill=green!20] {$\fs,\fs$};
  \node[main node] (2) [below left of=4, fill=orange!20] {$\es, \ec$};
  \node[main node] (1) [below right of=2, fill=blue!20, initial below] {$\ec, \ec$};
  \node[main node] (3) [below right of=4, fill=orange!20] {$\ec,\es$};
  \node[main node] (6) [right of=4, fill=green!20] {$\fs, \ec$};
  \node[main node] (5) [left of=4, fill=green!20] {$\ec, \fs$};

  \path[every node/.style={font=\sffamily\small}]
    (1) edge [bend left=30]  		(4)
        edge [bend left=30]  		(2)
        edge [bend right=20] 		(3)
        edge [loop above]    		(1)
    (2) edge [loop left, dashed]    	(2)
        edge [bend left=15]  		(4)
    (3) edge [loop right, dashed]   	(3)
        edge [bend right=15] 		(4)
    (4) edge [bend left=30]  		(1)
        edge [loop above]    		(4)
        edge [bend left=10]  		(5)
        edge [bend left=10]  		(6)
    (5) edge [loop left]     		(5)
        edge [bend right=75] 		(1)
        edge [bend left=10]  		(4)
        edge                 		(2)
    (6) edge [loop right=20] 		(6)
        edge [bend left=10]  		(4)
        edge [bend left=75]  		(1)
        edge                 		(3);
\end{tikzpicture}
\end{figure}
Nous allons, dans un premier temps ne regarder que deux exécutions à la fois et regarder quelles paires d'exécutions impliquent la non-T-diagnosticabilité. Dans la figure \ref{a2diag} est une abstraction de la twin machine $A\times A$ montrant toutes les transitions qui sont compatibles avec la T-diagnosticabilité. Chaque noeud contient un paire qui représente l'état de chaque exécution. Un $\ec$ représente un état sain, un $\es$ représente un état fautif qui n'a pas encore été détecté et un $\fs$ représente un état fautif qui a déjà été détecté. Une flèche pleine indique une transition possible et une flèche non-pleine indique une transition possible mais qui ne peut être prise qu'un nombe borné de fois jusqu'à-ce qu'on quitte l'état (ce qui veut dire que dans l'automate réel, il n'y a pas de boucle mais une chaîne finie d'états où le status ne change pas). Cet automate a été construit \`a partir des observations suivantes :
\begin{itemize}
	\item Un état sain peut subir une faute ($\ec \to \es$)
	\item On ne peut détecter une faute que si toutes les exécutions équivalentes observationnellement sont fautives (donc on ne peut obtenir des $\fs$ que en passant par un état ne contenant que des $\fs$, et donc par $(\fs,\fs)$ dans ce cas précis)
	\item Un état fautif dont la faute a été détectée peut se réparer ($\fs\to \ec$)
	\item Une faute doit être détectée en temps borné (donc on ne peut pas rester un temps non-borné dans un état contenant un $\es$)
\end{itemize}

Celà peut se généraliser à $k$ exécutions en utilisant ces mêmes observations. Plus formellement, les états sont des éléments de $\{\ec, \es, \fs \}^k$. Une transition de $q=(q_1, \dots, q_k)$ vers $q'=(q_1',\dots,q_k')$ est possible ssi $$\forall i \in \{1, \dots, k\}, (q_i = q_i')\lor ((q_i = \ec) $$$$\land (q_i'=\es)) \lor ((q_i = \fs)\land (q_i' = \ec))$$
\`a une exception près : si $q_1'=\dots=q_k'=\es$, alors la transition se fait en fait vers $(\fs, \dots, \fs)$ au lieu de se faire vers $(\es, \dots, \es)$. Toutes les transitions peuvent être prises un nombre non-borné de fois sans quitter l'état sauf celles telles que $q= q'$ et $\exists i\in \{1, \dots, k\}, q_i = \es = q_i'$. On appelera cet automate $B_k$.

Soit $A$ un automate avec états fautifs / non-fautifs. On pose $A\times B_k$ l'automate dont les états sont des paires d'états ayant les mêmes status (c'est-à-dire que pour chaque état $(a, b)$, $a_i$ fautif $\iff b_i$ fautif) et tel que $(a_1,b_1)\overset{x}{\to}(a_2, b_2)$ ssi $a_1\overset{x}{\to} a_2$ et $b_1 \to b_2$. On dira que $B_k$ modélise $A^k$ si pour chaque transition accessible $a_1\overset{x}{\to} a_2$ dans $A$, il existe $b_1$ et $b_2$ tels qu'il y ait une transition correspondante $(a_1,b_1)\overset{x}{\to}(a_2, b_2)$ (ce qui assure que $A^k$ n'essaye pas de prendre des transitions interdites dans $B_k$) et si $A\times B_k$ ne contient pas de cycle accessible dans lequel une composante reste à $\es$ (ce qui assure que toute faute est détectée en temps fini). 

Tester cette propriété peut se faire en espace polynomial par rapport à la taille de $A$ car il suffit de regarder des chemins de taille inférieure ou égale à $n$ (le nombre d'états dans $A$). Pour vérifier qu'aucune transition interdite n'est utilisée, on étudie les noeuds un par un. On commence par différencier les sous-états fautifs qui peuvent être des $\es$ ou des $\fs$, selon le chemin choisi, de ceux qui sont toujours des $\fs$ car ils ont plus de transitions possibles (dans le cas $k=2$, ce serait par exemple les transitions $(\ec,\fs)\to (\es, \ec)$ qui sont permises alors que $(\ec,\es)\to (\es, \ec)$ ne l'est pas), ce qui se fait en espace polynomial en $n$ en regardant les chemins de taille $n$ arrivant sur le nœud. La vérification que les transitions sortantes sont bien autorisées ou non découlent directement de la distinction $\es$ / $\fs$. La détection de cycle se fait en testant tous les cycles potentiels de taille inférieure ou égale à $n$ et en vérifiant à chaque fois si c'est effectiement un cycle et que chacun de se nœuds peut effectivement avoir un $\es$ en $i^\text{ème}$ composante (pour $i$ fixé et on itère ensuite pour chaque $i$ possible). Pour ce qui est de l'accessibilité, elle se décide également en regardant les chemins de tailles $n$ arrivant au nœud.

\begin{mydef}[$T_k$-diagnosticabilité]
Un automate $A$ est $T_k$-diagnosticable ssi $B_k$ modélise $A^k$.
\end{mydef}

Intuitivement, un automate est $T_k$-diagnosticable s'il est $T$-diagnosticable où si $k$ exécution ne suffisent pas à déterminer qu'il est non T-diagnosticable.

\begin{myth}[Conditions nécessaires de $T$-diagnosticabilité]
Soit $A$ un automate. Si $A$ est $T$-diagnosticable alors pour tout $k,$ $A$ est $T_k$-diagnosticable.
\end{myth}

\begin{myth}[Conditions suffisantes de $T$-diagnosticabilité]
Soit $A$ un automate \`a $n$ états. Pour tout $k \ge n$, si $A$ est $T_k$-diagnosticable, alors $A$ est $T$-diagnosticable.
\end{myth}

Intuitivement, la borne est à $n$ car pour $k\ge n$, $A^k$ peut simuler $\text{det}(A)$ en représentant un ensemble d'état de $A$ par sa tuple (où il y a en plus un ordre et des répétitions). 

\begin{myco}
Soit $A$ un automate \`a $n$ états. $A$ est $T$-diagnosticable ssi $A$ est $T_n$-diagnosticable.
\end{myco}

\begin{myth}
Le problème de décision de la T-diagnosticabilité est PSPACE.
\end{myth}

\begin{figure}
\caption{Exemple d'automate $T_2$-diagnosticable mais non $T_3$-diagnosticable (et donc non $T$-diagnosticable) par Eric Fabre}
\label{t2diag}
$D_{1,3}$ : 
\scalebox{0.95}{
\begin{tikzpicture}[->,>=stealth',shorten >=1pt,auto,node distance=1cm,
  thick,main node/.style={circle,draw,font=\sffamily\bfseries}, align=center]

  \baselineskip=0.1cm
  \lineskiplimit=4cm
  \lineskip=0.1cm

  \node[main node, initial, initial text={}] (0) {0};
  \node[main node, shape = rectangle, right of = 0] (1) {1};
  \node[main node, shape = rectangle, right of = 1] (2) {2};
  \node[main node, shape = rectangle, right of = 2] (3) {3};
  \node[main node, right of = 3] (4) {4};
  \node[main node, right of = 4] (5) {5};
  \node[main node, right of = 5] (6) {6};
  \node[main node, shape = rectangle, right of = 6] (7) {7};

  \path[every node/.style={font=\sffamily\small}]
    (0) edge (1)
    (1) edge (2)
    (2) edge (3)
    (3) edge (4)
    (4) edge (5)
    (5) edge (6)
    (6) edge (7)
    (7) edge [bend right = 30] (2);
\end{tikzpicture}
}
$D_{2,3}$ : 
\scalebox{0.95}{
\begin{tikzpicture}[->,>=stealth',shorten >=1pt,auto,node distance=1cm,
  thick,main node/.style={circle,draw,font=\sffamily\bfseries}, align=center]

  \baselineskip=0.1cm
  \lineskiplimit=4cm
  \lineskip=0.1cm

  \node[main node, initial, initial text={}] (0) {0};
  \node[main node, shape = rectangle, right of = 0] (1) {1};
  \node[main node, right of = 1] (2) {2};
  \node[main node, shape = rectangle, right of = 2] (3) {3};
  \node[main node, shape = rectangle, right of = 3] (4) {4};
  \node[main node, shape = rectangle, right of = 4] (5) {5};
  \node[main node, right of = 5] (6) {6};
  \node[main node, right of = 6] (7) {7};

  \path[every node/.style={font=\sffamily\small}]
    (0) edge (1)
    (1) edge (2)
    (2) edge (3)
    (3) edge (4)
    (4) edge (5)
    (5) edge (6)
    (6) edge (7)
    (7) edge [bend right = 30] (2);
\end{tikzpicture}
}
$D_{3,3}$ : 
\scalebox{0.95}{
\begin{tikzpicture}[->,>=stealth',shorten >=1pt,auto,node distance=1cm,
  thick,main node/.style={circle,draw,font=\sffamily\bfseries}, align=center]

  \baselineskip=0.1cm
  \lineskiplimit=4cm
  \lineskip=0.1cm

  \node[main node, initial, initial text={}] (0) {0};
  \node[main node, shape = rectangle, right of = 0] (1) {1};
  \node[main node, right of = 1] (2) {2};
  \node[main node, right of = 2] (3) {3};
  \node[main node, right of = 3] (4) {4};
  \node[main node, shape = rectangle, right of = 4] (5) {5};
  \node[main node, shape = rectangle, right of = 5] (6) {6};
  \node[main node, shape = rectangle, right of = 6] (7) {7};

  \path[every node/.style={font=\sffamily\small}]
    (0) edge (1)
    (1) edge (2)
    (2) edge (3)
    (3) edge (4)
    (4) edge (5)
    (5) edge (6)
    (6) edge (7)
    (7) edge [bend right = 30] (2);
\end{tikzpicture}
}
\end{figure}

\begin{figure}
\caption{Exemple d'automate $T_3$-diagnosticable mais non $T_4$-diagnosticable (et donc non $T$-diagnosticable)}
\label{t3diag}
$D_{1,4}$ :
\scalebox{0.9}{
 \begin{tikzpicture}[->,>=stealth',shorten >=1pt,auto,node distance=0.8cm,
  thick,main node/.style={circle,draw,font=\sffamily\bfseries}, align=center]

  \baselineskip=0.1cm
  \lineskiplimit=4cm
  \lineskip=0.1cm

  \node[main node, initial, initial text={}] (0) {0};
  \node[main node, shape = rectangle, right of = 0] (1) {1};
  \node[main node, shape = rectangle, right of = 1] (2) {2};
  \node[main node, shape = rectangle, right of = 2] (3) {3};
  \node[main node, shape = rectangle, right of = 3] (4) {4};
  \node[main node, shape = rectangle, right of = 4] (5) {5};
  \node[main node, right of = 5] (6) {6};
  \node[main node, right of = 6] (7) {7};
  \node[main node, right of = 7] (8) {8};
  \node[main node, shape = rectangle, right of = 8] (9) {9};

  \path[every node/.style={font=\sffamily\small}]
    (0) edge (1)
    (1) edge (2)
    (2) edge (3)
    (3) edge (4)
    (4) edge (5)
    (5) edge (6)
    (6) edge (7)
    (7) edge (8)
    (8) edge (9)
    (9) edge [bend right = 30] (2);
\end{tikzpicture}
}
$D_{2,4}$ :
\scalebox{0.9}{
 \begin{tikzpicture}[->,>=stealth',shorten >=1pt,auto,node distance=0.8cm,
  thick,main node/.style={circle,draw,font=\sffamily\bfseries}, align=center]

  \baselineskip=0.1cm
  \lineskiplimit=4cm
  \lineskip=0.1cm

  \node[main node, initial, initial text={}] (0) {0};
  \node[main node, shape = rectangle, right of = 0] (1) {1};
  \node[main node, right of = 1] (2) {2};
  \node[main node, shape = rectangle, right of = 2] (3) {3};
  \node[main node, shape = rectangle, right of = 3] (4) {4};
  \node[main node, shape = rectangle, right of = 4] (5) {5};
  \node[main node, shape = rectangle, right of = 5] (6) {6};
  \node[main node, shape = rectangle, right of = 6] (7) {7};
  \node[main node, right of = 7] (8) {8};
  \node[main node, right of = 8] (9) {9};

  \path[every node/.style={font=\sffamily\small}]
    (0) edge (1)
    (1) edge (2)
    (2) edge (3)
    (3) edge (4)
    (4) edge (5)
    (5) edge (6)
    (6) edge (7)
    (7) edge (8)
    (8) edge (9)
    (9) edge [bend right = 30] (2);
\end{tikzpicture}
}
$D_{3,4}$ :
\scalebox{0.9}{
 \begin{tikzpicture}[->,>=stealth',shorten >=1pt,auto,node distance=0.8cm,
  thick,main node/.style={circle,draw,font=\sffamily\bfseries}, align=center]

  \baselineskip=0.1cm
  \lineskiplimit=4cm
  \lineskip=0.1cm

  \node[main node, initial, initial text={}] (0) {0};
  \node[main node, shape = rectangle, right of = 0] (1) {1};
  \node[main node, right of = 1] (2) {2};
  \node[main node, right of = 2] (3) {3};
  \node[main node, right of = 3] (4) {4};
  \node[main node, shape = rectangle, right of = 4] (5) {5};
  \node[main node, shape = rectangle, right of = 5] (6) {6};
  \node[main node, shape = rectangle, right of = 6] (7) {7};
  \node[main node, shape = rectangle, right of = 7] (8) {8};
  \node[main node, shape = rectangle, right of = 8] (9) {9};

  \path[every node/.style={font=\sffamily\small}]
    (0) edge (1)
    (1) edge (2)
    (2) edge (3)
    (3) edge (4)
    (4) edge (5)
    (5) edge (6)
    (6) edge (7)
    (7) edge (8)
    (8) edge (9)
    (9) edge [bend right = 30] (2);
\end{tikzpicture}
}
$D_{3,4}$ : 
\scalebox{0.9}{
\begin{tikzpicture}[->,>=stealth',shorten >=1pt,auto,node distance=0.8cm,
  thick,main node/.style={circle,draw,font=\sffamily\bfseries}, align=center]

  \baselineskip=0.1cm
  \lineskiplimit=4cm
  \lineskip=0.1cm

  \node[main node, initial, initial text={}] (0) {0};
  \node[main node, shape = rectangle, right of = 0] (1) {1};
  \node[main node, shape = rectangle, right of = 1] (2) {2};
  \node[main node, shape = rectangle, right of = 2] (3) {3};
  \node[main node, right of = 3] (4) {4};
  \node[main node, right of = 4] (5) {5};
  \node[main node, right of = 5] (6) {6};
  \node[main node, shape = rectangle, right of = 6] (7) {7};
  \node[main node, shape = rectangle, right of = 7] (8) {8};
  \node[main node, shape = rectangle, right of = 8] (9) {9};

  \path[every node/.style={font=\sffamily\small}]
    (0) edge (1)
    (1) edge (2)
    (2) edge (3)
    (3) edge (4)
    (4) edge (5)
    (5) edge (6)
    (6) edge (7)
    (7) edge (8)
    (8) edge (9)
    (9) edge [bend right = 30] (2);
\end{tikzpicture}
}
\end{figure}

On notera que la condition $k\ge n$ est imporante. Un contre exemple pour $k<n$ est donné dans la figure \ref{t2diag}. En effet, seule la première faut est diagnostiquée car ensuite, les fautes s'enchaînent mais les trois séquences ne sont jamais fautives en même temps et on ne peut donc jamais détecter la faute. Cependant, si l'on ne considère que deux des sous-automates, alors le résultat est $T$-diagnosticable. D'ailleurs, l'automate composé des trois sous-automates est $T_2$-diagnosticable mais pas $T_3$-diagnosticable. Note : L'absence de lettres sur les transitions est due au fait qu'elles n'apportent rien. On pourrait autoriser toutes les lettres sur toutes les transitions ou choisir une lettre et ne mettre qu'elle partout, le résultat serait le même. Un autre exemple est donné dans la figure \ref{t3diag}. On voit assez clairement comment construire les sous-automates qui génèrent chacun une seule séquence : On comment par un état non-fautif suivi d'un état fautif puis on met $2n$ états en chaine (et dont le dernier état pointe vers le $3^{eme}$ état) pour construire $D_{k,n}$. Parmis ces $2n$ états, $3$ sont non-fautifs et le reste est fautif. Pour passer de $D_{k,n}$ à $D_{k+1,n}$, on déplace simplement les trois états fautifs de deux états vers la droite, modulo le cycle. L'automate formé de $D_{1,n},\dots,D_{n,n}$ est $T_{n-1}$-diagnosticable mais pas $T_n$-diagnosticable.

\begin{myconj}
Le problème de décision de la T-diagnosticabilité est PSPACE-dur.
\end{myconj}

\begin{mydis}
On exhibe une réduction polynomiale du problème de la vacuité de l'intersection de $k$ langages rationnels dont ont sait qu'il est PSPACE-complet \cite{Lange92theemptiness}. Soit $L_1, \dots, L_k$ des langages rationnels. Soient $D_{1,k},\dots, D_{k,k}$ les automates construits précédement. Soit $\alpha$ une lettre n'apparaissant dans l'aphabet d'aucun des $L_i$. Soit $A$ l'automate consistant de $L_1, \dots, L_k, D_{1,k},\dots, D_{k,k}$ avec des transitions étiquettées par $\alpha$ des états finaux de $L_i$ à l'état initial de $D_{i,k}$, et ce pour chaque $i$. Les états initiaux de $A$ sont les états initiaux des $L_i$. Alors $A$ est $T$-diagnosticable ssi l'intersection des $L_i$ est vide. L'idée est que si l'intersection est vide alors pour tout mot, il y aura au moins un des $D_{i,k}$ que l'on ne pourra pas atteindre et on sera donc $T$-diagnosticable. Et si l'intersection des $L_i$ est non-vide alors après avoir lu ce mot, un $\alpha$, on peut entrer dans tous les $D_{i,k}$ et le résultat n'est donc pas $T$-diagnosticable. On notera que le $\alpha$ sert a synchroniser les différents morceaux de l'automate (dans le sente où lorsque l'on va construire $A^k$, on pourra soit lire alpha partout, soit ne pas le lire, ce qui évite que pour certains $i$, le mot lu par $L_i$ soit plus court que pour d'autres (un préfixe en fait), ce qui aurait pour effet de fausser le résultat et de déphaser les $D_{i,k}$.
\end{mydis}

En admettant cette conjecture (dont nous sommes persuadés mais n'en ayant pas encore trop discuté avec nos encadrants, nous préférons rester prudents) on peut prouver que la T-diagnosticabilité est PSPACE-complet.
\begin{myth}
Si la conjecture 1 est vraie alors le problème de décision de la T-diagnosticabilité est PSPACE-complet.
\end{myth}
\begin{proof}
T-diagnosticabilité est PSPACE car nous avons donné un algorithme poylnomial en mémoire et T-diagnosticabilité est PSPACE-dur d'après la conjecture 1.
\end{proof}



% An example of a floating figure using the graphicx package.
% Note that \label must occur AFTER (or within) \caption.
% For figures, \caption should occur after the \includegraphics.
% Note that IEEEtran v1.7 and later has special internal code that
% is designed to preserve the operation of \label within \caption
% even when the captionsoff option is in effect. However, because
% of issues like this, it may be the safest practice to put all your
% \label just after \caption rather than within \caption{}.
%
% Reminder: the "draftcls" or "draftclsnofoot", not "draft", class
% option should be used if it is desired that the figures are to be
% displayed while in draft mode.
%
%\begin{figure}[!t]
%\centering
%\includegraphics[width=2.5in]{myfigure}
% where an .eps filename suffix will be assumed under latex, 
% and a .pdf suffix will be assumed for pdflatex; or what has been declared
% via \DeclareGraphicsExtensions.
%\caption{Simulation Results}
%\label{fig_sim}
%\end{figure}

% Note that IEEE typically puts floats only at the top, even when this
% results in a large percentage of a column being occupied by floats.


% An example of a double column floating figure using two subfigures.
% (The subfig.sty package must be loaded for this to work.)
% The subfigure \label commands are set within each subfloat command, the
% \label for the overall figure must come after \caption.
% \hfil must be used as a separator to get equal spacing.
% The subfigure.sty package works much the same way, except \subfigure is
% used instead of \subfloat.
%
%\begin{figure*}[!t]
%\centerline{\subfloat[Case I]\includegraphics[width=2.5in]{subfigcase1}%
%\label{fig_first_case}}
%\hfil
%\subfloat[Case II]{\includegraphics[width=2.5in]{subfigcase2}%
%\label{fig_second_case}}}
%\caption{Simulation results}
%\label{fig_sim}
%\end{figure*}
%
% Note that often IEEE papers with subfigures do not employ subfigure
% captions (using the optional argument to \subfloat), but instead will
% reference/describe all of them (a), (b), etc., within the main caption.


% An example of a floating table. Note that, for IEEE style tables, the 
% \caption command should come BEFORE the table. Table text will default to
% \footnotesize as IEEE normally uses this smaller font for tables.
% The \label must come after \caption as always.
%
%\begin{table}[!t]
%% increase table row spacing, adjust to taste
%\renewcommand{\arraystretch}{1.3}
% if using array.sty, it might be a good idea to tweak the value of
% \extrarowheight as needed to properly center the text within the cells
%\caption{An Example of a Table}
%\label{table_example}
%\centering
%% Some packages, such as MDW tools, offer better commands for making tables
%% than the plain LaTeX2e tabular which is used here.
%\begin{tabular}{|c||c|}
%\hline
%One & Two\\
%\hline
%Three & Four\\
%\hline
%\end{tabular}
%\end{table}


% Note that IEEE does not put floats in the very first column - or typically
% anywhere on the first page for that matter. Also, in-text middle ("here")
% positioning is not used. Most IEEE journals/conferences use top floats
% exclusively. Note that, LaTeX2e, unlike IEEE journals/conferences, places
% footnotes above bottom floats. This can be corrected via the \fnbelowfloat
% command of the stfloats package.

\section{Conclusion}
Dans ce papier, nous avons généralisé la notion classique de diagnosticabilit\'e en autorisant les fautes intermitentes. Nous 
avons trouv\'e un algorithme PSPACE pour ce probl\`eme. Nous avons expliquer pouquoi nous pensons que ce problème est PSPACE-complet. Nous cherchons maintenant à prouver rigoureusement cette conjecture.




% conference papers do not normally have an appendix


% use section* for acknowledgement
\section*{Acknowledgment}


TODO The authors would like to thank...





% trigger a \newpage just before the given reference
% number - used to balance the columns on the last page
% adjust value as needed - may need to be readjusted if
% the document is modified later
%\IEEEtriggeratref{8}
% The "triggered" command can be changed if desired:
%\IEEEtriggercmd{\enlargethispage{-5in}}

% references section

% can use a bibliography generated by BibTeX as a .bbl file
% BibTeX documentation can be easily obtained at:
% http://www.ctan.org/tex-archive/biblio/bibtex/contrib/doc/
% The IEEEtran BibTeX style support page is at:
% http://www.michaelshell.org/tex/ieeetran/bibtex/
%\bibliographystyle{IEEEtran}
% argument is your BibTeX string definitions and bibliography database(s)
%\bibliography{IEEEabrv,../bib/paper}
%
% <OR> manually copy in the resultant .bbl file
% set second argument of \begin to the number of references
% (used to reserve space for the reference number labels box)
\bibliographystyle{plain}
\bibliography{bibliography}




% that's all folks
\end{document}


