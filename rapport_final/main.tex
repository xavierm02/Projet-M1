\documentclass[10pt,a4paper]{article}
\usepackage[utf8]{inputenc}
\usepackage[francais]{babel}
\usepackage[T1]{fontenc}
\usepackage{amsmath}
\usepackage{amsfonts}
\usepackage{amssymb}
\author{Fran\c{c}ois \textsc{Godi}, Chen \textsc{Qian} et Xavier \textsc{Montillet}}
\title{T-diagnosticabilit\'e}
\begin{document}
\maketitle

\begin{abstract}
La notion classique de diagnosticabilit\'e ne permet de mod\'eliser que des fautes permanentes, et le langage des fautes est donc suppos\'e suffixe-clos. Dans ce papier, nous rel\^achons cette hypoth\`ese afin de permettre la mod\'elisation de fautes r\'eparables. Pour ce faire, nous introduisons la T-diagnosticabilit\'e qui formalise le fait que toute faute est d\'etect\'ee en temps born\'e et avant d'\^etre r\'epar\'ee.

Nous montrons que la T-diagnosticabilit\'e \'etend bien la notion de diagnosticabilit\'e, c'est-\`a-dire que pour un syst\`eme dont les fautes sont permanentes, les notions de T-diagnosticabilit\'e et de diagnosticabilit\'e coincident.

Nous proposons un algorithme quadratique de d\'ecision pour la T-diagnosticabilit\'e et une construction de R-diagnostiqueur.
\end{abstract}

\section{Introduction}

\section{Pr\'erequis}

Automates

Diagnosticabilit\'e

R\'eduction : Langage de fautes -> \'etats fautifs / non-fautifs

R\'eduction aux mots fautifs minimaux

Simulation

\section{T-Diagnosticabilit\'e}

Diag classique

Decidabilit\'e, Twin machine

T-diagnosticabilit\'e, Transient fault diagnosticability

T-diag sans r\'eparation <-> diag

\section{$\mathcal{A}$-diagnosticabilit\'e}

automate diag, equivalence

automate T-diag, equivalence

remarque T-diag sans r\'eparation <-> diag

\subsection{Diagnostiqueur}

diagnostiqueur

\end{document}