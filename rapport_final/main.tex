\documentclass[10pt,a4paper]{article}
\usepackage[utf8]{inputenc}
\usepackage[francais]{babel}
\usepackage[T1]{fontenc}
\usepackage{amsmath}
\usepackage{amsfonts}
\usepackage{amssymb}
\usepackage{tikz}
\usepackage{wasysym}
\newtheorem{mydef}{D\'efinition}
\newtheorem{myth}{Th\'eor\`eme}

\usepackage{amssymb}\usepackage{varwidth}
\usetikzlibrary{arrows,positioning,automata,shadows}

\newcommand{\es}{\square}
\newcommand{\fs}{\blacksquare}
\newcommand{\ec}{\ocircle}

\author{Fran\c{c}ois \textsc{Godi}, Chen \textsc{Qian} et Xavier \textsc{Montillet}}
\title{T-diagnosticabilit\'e}
\begin{document}
\maketitle

\begin{abstract}
La notion classique de diagnosticabilit\'e ne permet de mod\'eliser que des fautes permanentes, et le langage des fautes est donc suppos\'e suffixe-clos. Dans ce papier, nous rel\^achons cette hypoth\`ese afin de permettre la mod\'elisation de fautes r\'eparables. Pour ce faire, nous introduisons la T-diagnosticabilit\'e qui formalise le fait que toute faute est d\'etect\'ee en temps born\'e et avant d'\^etre r\'epar\'ee.

Nous montrons que la T-diagnosticabilit\'e \'etend bien la notion de diagnosticabilit\'e, c'est-\`a-dire que pour un syst\`eme dont les fautes sont permanentes, les notions de T-diagnosticabilit\'e et de diagnosticabilit\'e coincident.

Nous proposons un algorithme quadratique de d\'ecision pour la T-diagnosticabilit\'e et une construction de R-diagnostiqueur.
\end{abstract}

\section*{Introduction}

De nos jours, les syst\`emes d'information ont tendance \`a \^etre trop complexes pour que l'on puisse se convaincre de leur sûret\'e. On les mod\'elise donc formellement pour prouver des propriétés. La diagnosticabilit\'e est une propri\'et\'e qui exprime la capacit\'e \`a d\'etecter les fautes du syst\`eme \`a partir d'une connaissance partielle de son \'etat.

La notion classique de diagnosticabilit\'e\cite{SamSRST96} ne s'applique qu'aux systèmes où les fautes sont définitives. Dans ce papier, nous généralisons cette notion afin de prendre en compte les systèmes où les fautes peuvent être r\'eparées. Notre nouvelle notion, la R-diagnosticabilit\'e, formalise le fait que toute faute est détectée en temps born\'e et avant d'avoir été r\'epar\'ee.

Après avoir rappellé les prérequis dans la section I, la définition classique de la diagnosticabilité dans la section II et introduit la définition de la R-diagnosticabilit\'e dans la section III, nous proposerons dans la section IV un R-diagnostiqueur et un algorithme de d\'ecision quadratique pour la R-diagnosticabilit\'e. Nous montrerons ensuite dans la section V que la R-diagnosticabilit\'e \'etend bien la notion de diagnosticabilit\'e, c'est-\`a-dire que pour un syst\`eme dont les fautes sont permanentes, la R-diagnosticabilit\'e et la diagnosticabilit\'e sont \'equivalentes.

\newpage

\section{Pr\'erequis}

Automates

Diagnosticabilit\'e

R\'eduction : Langage de fautes $\Rightarrow$ \'etats fautifs / non-fautifs

R\'eduction aux mots fautifs minimaux

Simulation

\section{T-Diagnosticabilit\'e}

Diag classique

Decidabilit\'e, Twin machine

T-diagnosticabilit\'e, Transient fault diagnosticability

T-diag sans r\'eparation $ \iff $ diag

\subsection*{[Temporaire] Defs}

Soit $S$ un système dont les évènements sont dans $\Sigma = \Sigma_o \ \bigsqcup \Sigma_{no}$
\paragraph{T-diag (1) :}

\begin{multline*}
\forall u \in L_F(S).\ \exists n \in \mathbb{N}.\ \forall v \in \Sigma^*.\ \exists v' \in \Sigma^*. \\
(uv \in L(S) \ \wedge \  ||v||_{\Sigma_o}\geq n) \Rightarrow (v'\leq v \ \wedge \  u.Pref(v') \subseteq L_F(S) \ \wedge \  [uv'] \cap L(S) \subseteq L_F(S))
\end{multline*}


\paragraph{Langage de faute minimal :}
$$L_{FMIN}(S) = \{u \in L_F(S) \mid \forall v,w \in \Sigma^*, u = vw \land v.\text{Pref}(w)\subseteq L_F(S) \implies w = \varepsilon \}$$
ou bien
$$L_{FMIN}(S) = \{u \in L_F(S) \mid \forall a \in \Sigma, \forall v \in \Sigma^*, u = va \implies v \not \in L_{FMIN}(S)\}$$
\paragraph{T-diag (2) :}
\begin{multline*}
\forall u \in L_{FMIN}(S).\ \exists n \in \mathbb{N}.\ \forall v \in \Sigma^*.\ \exists v' \in \Sigma^*. \\
(uv \in L(S) \ \wedge \  ||v||_{\Sigma_o}\geq n) \Rightarrow (v'\leq v \ \wedge \  u.Pref(v') \subseteq L_F(S) \ \wedge \  [uv'] \cap L(S) \subseteq L_F(S))
\end{multline*}


\newpage
\section{Décision de la T-diagnosticabilité}
\begin{figure}
\caption{Description des paires d'excécutions compatibles avec la T-diagnosticabilité}
\label{a2diag}
\begin{tikzpicture}[->,>=stealth',shorten >=1pt,auto,node distance=4cm,
  thick,main node/.style={circle,draw,font=\sffamily\Large\bfseries}, align=center, minimum size=1.4cm]

  \baselineskip=0.1cm
  \lineskiplimit=4cm
  \lineskip=0.1cm

  \node[main node] (4) [fill=green!20] {$\fs,\fs$};
  \node[main node] (2) [below left of=4, fill=orange!20] {$\es, \ec$};
  \node[main node] (1) [below right of=2, fill=blue!20, initial below] {$\ec, \ec$};
  \node[main node] (3) [below right of=4, fill=orange!20] {$\ec,\es$};
  \node[main node] (6) [right of=4, fill=green!20] {$\fs, \ec$};
  \node[main node] (5) [left of=4, fill=green!20] {$\ec, \fs$};

  \path[every node/.style={font=\sffamily\small}]
    (1) edge [bend left=30]  		(4)
        edge [bend left=30]  		(2)
        edge [bend right=20] 		(3)
        edge [loop above]    		(1)
    (2) edge [loop left, dashed]    	(2)
        edge [bend left=15]  		(4)
    (3) edge [loop right, dashed]   	(3)
        edge [bend right=15] 		(4)
    (4) edge [bend left=30]  		(1)
        edge [loop above]    		(4)
        edge [bend left=10]  		(5)
        edge [bend left=10]  		(6)
    (5) edge [loop left]     		(5)
        edge [bend right=75] 		(1)
        edge [bend left=10]  		(4)
        edge                 		(2)
    (6) edge [loop right=20] 		(6)
        edge [bend left=10]  		(4)
        edge [bend left=75]  		(1)
        edge                 		(3);
\end{tikzpicture}
\end{figure}
Nous allons, dans un premier temps ne regarder que deux exécutions à la fois et regarder quelles paires d'exécutions impliquent la non-T-diagnosticabilité. Dans la figure \ref{a2diag} est une abstraction de la twin machine $A\times A$ montrant toutes les transitions qui sont compatibles avec la T-diagnosticabilité. Chaque noeud contient un paire qui représente l'état de chaque exécution. Un $\ec$ représente un état sain, un $\es$ représente un état fautif qui n'a pas encore été détecté et un $\fs$ représente un état fautif qui a déjà été détecté. Une flèche pleine indique une transition possible et une flèche non-pleine indique une transition possible mais qui ne peut être prise qu'un nombe borné de fois jusqu'à-ce qu'on quitte l'état (ce qui veut dire que dans l'automate réel, il n'y a pas de boucle mais une chaîne finie d'états où le status ne change pas). Cet automate a été construit \`a partir des observations suivantes :
\begin{itemize}
	\item Un état sain peut subir une faute ($\ec \to \es$)
	\item On ne peut détecter une faute que si toutes les exécutions équivalentes observationnellement sont fautives (donc on ne peut obtenir des $\fs$ que en passant par un état ne contenant que des $\fs$, et donc par $(\fs,\fs)$ dans ce cas précis)
	\item Un état fautif dont la faute a été détectée peut se réparer ($\fs\to \ec$)
	\item Une faute doit être détectée en temps borné (donc on ne peut pas rester un temps non-borné dans un état contenant un $\es$)
\end{itemize}

Celà peut se généraliser à $k$ exécutions en utilisant ces mêmes observations. Plus formellement, les états sont des éléments de $\{\ec, \es, \fs \}^k$. Une transition de $q=(q_1, \dots, q_k)$ vers $q'=(q_1',\dots,q_k')$ est possible ssi $$\forall i \in \{1, \dots, k\}, (q_i = q_i')\lor ((q_i = \ec) \land (q_i'=\es)) \lor ((q_i = \fs)\land (q_i' = \ec))$$
\`a une exception près : si $q_1'=\dots=q_k'=\es$, alors la transition se fait en fait vers $(\fs, \dots, \fs)$ au lieu de se faire vers $(\es, \dots, \es)$. Toutes les transitions peuvent être prises un nombre non-borné de fois sans quitter l'état sauf celles telles que $q= q'$ et $\exists i\in \{1, \dots, k\}, q_i = \es = q_i'$. On appelera cet automate $B_k$.

Soit $A$ un automate avec états fautifs / non-fautifs. On pose $A\times B_k$ l'automate dont les états sont des paires d'états ayant les mêmes status (c'est-à-dire que pour chaque état $(a, b)$, $a_i$ fautif $\iff b_i$ fautif) et tel que $(a_1,b_1)\overset{x}{\to}(a_2, b_2)$ ssi $a_1\overset{x}{\to} a_2$ et $b_1 \to b_2$. On dira que $B_k$ modélise $A^k$ si pour chaque transition accessible $a_1\overset{x}{\to} a_2$ dans $A$, il existe $b_1$ et $b_2$ tels qu'il y ait une transition correspondante $(a_1,b_1)\overset{x}{\to}(a_2, b_2)$ (ce qui assure que $A^k$ n'essaye pas de prendre des transitions interdites dans $B_k$) et si $A\times B_k$ ne contient pas de cycle accessible dont tous les noeuds contiennt un $\es$ dans la composante venant de $B_k$ (ce qui oblige à détecter toute faute en temps borné). Tester cette propriété peut se faire en espace polynomial par rapport à la taille de $A$: 

def $T_i-diag$ : ca marche pour $i$ executions

T-diag -> $\forall i, T_i-diag$

$i\ge n \land T_i-diag \implies T-diag$


\section{$\mathcal{A}$-diagnosticabilit\'e}

automate diag, equivalence

automate T-diag, equivalence

remarque T-diag sans r\'eparation $ \iff $ diag

\begin{tikzpicture}[->,>=stealth',shorten >=1pt,auto,node distance=4cm,
  thick,main node/.style={circle,draw,font=\sffamily\Large\bfseries}, align=center, minimum size=1.4cm]

  \baselineskip=0.1cm
  \lineskiplimit=4cm
  \lineskip=0.1cm

  \node[main node] (4) [fill=green!20] {$\Box\Box$ \\ \\ \small $ $ \normalsize};
  \node[main node] (2) [below left of=4, fill=orange!20] {$\Box \ocircle$ \\ \small $0$ \normalsize};
  \node[main node] (1) [below right of=2, fill=blue!20] {$\ocircle \ocircle$ \\ \small $ $ \normalsize};
  \node[main node] (3) [below right of=4, fill=orange!20] {$\ocircle\Box$ \\ \small $0$ \normalsize};
  \node[main node] (6) [right of=4, fill=green!20] {$\Box \ocircle$ \\ \small $1$ \normalsize};
  \node[main node] (5) [left of=4, fill=green!20] {$\ocircle \Box$ \\ \small $1$ \normalsize};

  \path[every node/.style={font=\sffamily\small}]
    (1) edge [bend left=10]  		(4)
        edge [bend left=20]  		(2)
        edge [bend right=20] 		(3)
        edge [loop below]    		(1)
    (2) edge [loop left, dashed]    	(2)
        edge [bend left=15]  		(4)
    (3) edge [loop right, dashed]   	(3)
        edge [bend right=15] 		(4)
    (4) edge [bend left=10]  		(1)
        edge [loop above]    		(4)
        edge [bend left=10]  		(5)
        edge [bend left=10]  		(6)
    (5) edge [loop left]     		(5)
        edge [bend right=75] 		(1)
        edge [bend left=10]  		(4)
        edge                 		(2)
    (6) edge [loop right=20] 		(6)
        edge [bend left=10]  		(4)
        edge [bend left=75]  		(1)
        edge                 		(3);
\end{tikzpicture}

\section{Diagnostiqueur}

La construction du diagnostiqueur est juste de faire la d\'eterminisation de l'automate $\mathcal{A}_{2Etats}$. Pour utiliser ce diagnostiqueur, nous allons ex\'ecuter les mots sur ce diagnostiqueur. Quand nous arrivons dans un \'etat qui est compos\'e que par des \'etats $\Box$, nous allons lever un alarme pour indiquer qu'il y a eu une faute.

\section*{Conclusion}

Dans ce papier, nous avons poser un nouveau type de diagnosticabilit\'e, nous 
avons trouv\'e un algorithme PSPACE de ce probl\`eme. Cette notion extendre la diagnosticabilit\'e avec les fautes r\'eparables qui est souvent le cas dans notre vie courrante. La propri\'et\'e que nous nous avons int\'eress\'e est aussi applicable \`a des cas concr\`etes, par examples dans les syst\`emes qui peut r\'eparer des pannes tout seul, mais nous avons quand m\^eme envie de d\'etecter les fautes qu'il a eu lieu.
\bibliographystyle{plain}
\bibliography{bibliography}



\end{document}


