\documentclass[10pt,a4paper]{article}
\usepackage[utf8]{inputenc}
\usepackage[francais]{babel}
\usepackage[T1]{fontenc}
\usepackage{amsmath}
\usepackage{amsfonts}
\usepackage{amssymb}
\usepackage{tikz}
\usepackage{wasysym}
\usepackage{varwidth}
\usetikzlibrary{arrows,positioning,automata,shadows}
\author{Fran\c{c}ois \textsc{Godi}, Chen \textsc{Qian} et Xavier \textsc{Montillet}}
\title{T-diagnosticabilit\'e}
\begin{document}
\maketitle

\section*{Introduction}

De nos jours, les syst\`emes d'information ont tendance \`a \^etre trop complexes pour que l'on puisse se convaincre de leur sûret\'e. On les mod\'elise donc formellement pour prouver des propriétés. La diagnosticabilit\'e est une propri\'et\'e qui exprime la capacit\'e \`a d\'etecter les fautes du syst\`eme \`a partir d'une connaissance partielle de son \'etat.

La notion classique de diagnosticabilit\'e\cite{SamSRST96} ne s'applique qu'aux systèmes où les fautes sont définitives. Dans ce papier, nous généralisons cette notion afin de prendre en compte les systèmes où les fautes peuvent être r\'eparées. Notre nouvelle notion, la R-diagnosticabilit\'e, formalise le fait que toute faute est détectée en temps born\'e et avant d'avoir été r\'epar\'ee.

Après avoir rappellé les prérequis dans la section I, la définition classique de la diagnosticabilité dans la section II et introduit la définition de la R-diagnosticabilit\'e dans la section III, nous proposerons dans la section IV un R-diagnostiqueur et un algorithme de d\'ecision quadratique pour la R-diagnosticabilit\'e. Nous montrerons ensuite dans la section V que la R-diagnosticabilit\'e \'etend bien la notion de diagnosticabilit\'e, c'est-\`a-dire que pour un syst\`eme dont les fautes sont permanentes, la R-diagnosticabilit\'e et la diagnosticabilit\'e sont \'equivalentes.

\begin{tikzpicture}[->,>=stealth',shorten >=1pt,auto,node distance=4cm,
  thick,main node/.style={circle,draw,font=\sffamily\Large\bfseries}, align=center, minimum size=1.4cm]

  \baselineskip=0.1cm
  \lineskiplimit=4cm
  \lineskip=0.1cm

  \node[main node] (4) [fill=green!20] {$\Box\Box$ \\ \\ \small $ $ \normalsize};
  \node[main node] (2) [below left of=4, fill=orange!20] {$\Box \ocircle$ \\ \small $0$ \normalsize};
  \node[main node] (1) [below right of=2, fill=blue!20] {$\ocircle \ocircle$ \\ \small $ $ \normalsize};
  \node[main node] (3) [below right of=4, fill=orange!20] {$\ocircle\Box$ \\ \small $0$ \normalsize};
  \node[main node] (6) [right of=4, fill=green!20] {$\Box \ocircle$ \\ \small $1$ \normalsize};
  \node[main node] (5) [left of=4, fill=green!20] {$\ocircle \Box$ \\ \small $1$ \normalsize};

  \path[every node/.style={font=\sffamily\small}]
    (1) edge [bend left=10]  		(4)
        edge [bend left=20]  		(2)
        edge [bend right=20] 		(3)  
        edge [loop below]    		(1)
    (2) edge [loop left, dashed]    	(2)
        edge [bend left=15]  		(4)
    (3) edge [loop right, dashed]   	(3)
        edge [bend right=15] 		(4)  
    (4) edge [bend left=10]  		(1)
        edge [loop above]    		(4)
	edge [bend left=10]  		(5)
	edge [bend left=10]  		(6)
    (5) edge [loop left]     		(5)
	edge [bend right=75] 		(1)
	edge [bend left=10]  		(4)
   	edge                 		(2)
    (6) edge [loop right=20] 		(6)	
 	edge [bend left=10]  		(4)
	edge [bend left=75]  		(1)
   	edge                 		(3);
\end{tikzpicture}


\bibliographystyle{plain}
\bibliography{bibliography}

\end{document}
