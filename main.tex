\documentclass[10pt,a4paper]{article}
\usepackage[utf8]{inputenc}
\usepackage[francais]{babel}
\usepackage[T1]{fontenc}
\usepackage{amsmath}
\usepackage{amsfonts}
\usepackage{amssymb}
\usepackage{mathtools}
\usepackage{stmaryrd}
\usepackage{tikz}
\usepackage{float}
\newtheorem{mydef}{D\'efinition}
\newtheorem{myth}{Th\'eor\`eme}
\newcommand{\enum}[2]{\llbracket #1, #2 \rrbracket}
\usetikzlibrary{arrows}
\bibliographystyle{plain}

\begin{document}


    \title{Bibliographie sur l'opacit\'e}

    \author{François Godi, Xavier Montillet, Chen Qian}

\maketitle

%\begin{abstract}
%TODO 
%\end{abstract}

\section*{Introduction}

De nos jours, les syst\`emes d'information ont tendance \`a \^etre trop complexes pour que l'on puisse se convaincre de leur sûret\'e. On les mod\'elise donc formellement pour prouver des propriétés.  Ce rapport d\'ecrit les propri\'et\'es de diagnosticabilit\'e qui expriment la capacit\'e \`a d\'etecter les fautes du syst\`eme \`a partir d'une information partielle sur son \'etat, les propri\'et\'es de non-interf\'erence qui mod\'elisent le fait que les actions de certains utilisateurs du syst\`eme sont cach\'ees aux autres, et les propri\'et\'es d'opacit\'e qui expriment l'impossibilit\'e d'\^etre s\^ur qu'une ex\'ecution est dans un langage donn\'e.

\section{Définitions}

\subsection{Automates}

Un automate (avec $\varepsilon$-transitions) est un 5-uplet $A = (Q, \Sigma, \Delta, q_0, F)$ o\`u
\begin{itemize}
	\item $Q$ est l'ensemble des états du système.
	\item $\Sigma$ est l'ensemble des événements.
	\item $\Delta \subseteq (Q \times (\Sigma \cup \{\varepsilon\}) \times Q)$ est une relation de transition.
	\item $q_0 \in Q$ l'état initial.
	\item $F \subset Q$ l'ensemble des \'etats finaux.
\end{itemize}

Un automate fini est un automate tel que $Q$ est fini.

\'Etant donn\'es $p,q \in Q$ et $l\in \Sigma \cup \{\varepsilon\}$, on notera $p \underset{\Delta}{\overset{l}{{\to}}}q$ pour $(p,l,q) \in \Delta$. On notera \'egalement $q_0 \underset{\Delta}{\overset{l_0}{{\to}}} \dots \underset{\Delta}{\overset{l_n}{{\to}}} q_{n+1}$ pour $\forall k \in \enum{0}{n}, q_k 
\underset{\Delta}{\overset{l_k}{{\to}}} q_{k+1}$. Pour simplifier, le $\Delta$ sera omis lorsqu'il est \'evident.



On dit qu'un automate fini est sans $\varepsilon$-transitions si $\Delta \subseteq (Q\times \Sigma \times Q)$.

Un automate fini d\'eterministe est un automate fini sans $\varepsilon$-transitions tel que $\Delta$ peut \^etre vue comme une fonction de $Q\times \Sigma$ dans $Q$: $\forall p,q,q' \in Q, \forall l \in \Sigma, p\overset{l}{{\to}}q \land p\overset{l}{{\to}}q' \implies q=q'$

\subsection{Langage reconnu}

Un mot $l_1...l_n \in \left(\Sigma\cup \{\varepsilon\}\right)^*$ est reconnu par un automate $A$ si il existe $n \in \mathbb N$ et $q_1,...,q_n \in Q$ tel que $q_0 \overset{l_1}{{\to}} q_1 \overset{l_2}{{\to}} \dots \overset{l_n}{{\to}} q_{n+1}$ et $q_{n+1} \in F$.

Le langage $L(A)$ reconnu par un automate $A$ est l'ensemble des mots de $\Sigma^*$ reconnus par $A$.

\subsection{Langage extension clos}

Un langage $L$ est dit extension-clos si $\forall u \in L, \forall v \in \Sigma^*, uv \in L$.

\subsection{Produit synchrone}

Soient $A = (Q, \Sigma, \Delta, q_0, F)$ et $A' = (Q', \Sigma', \Delta', q_0', F')$ deux automates sans $\varepsilon$. On définit le produit synchrone de ces automates comme \'etant $$A\times A' = (Q \times Q', \Sigma \cap \Sigma', \Delta'', (q_0, q_0'), F\times F')$$ o\`u $\Delta''$ est d\'efinie par $(p,p')\overset{l}{{\to}}(q,q')$ si et seulement si $p\overset{l}{{\to}}q$ et $p'\overset{l}{{\to}}q'$.

On notera que $L(A\times B) = L(A) \cap L(B)$.

\subsection{$\varepsilon$-clôture}

On définit  $\operatorname{cl}_\varepsilon(A) = (Q, \Sigma, \Delta', q_0, F)$ comme l'$\varepsilon$-clôture \`a gauche de l'automate $A$, c'est-\`a-dire $p \underset{\Delta'}{\overset{l}{{\to}}} r$ si et seulement si il existe $n\ge 1$ et $q_1,\dots,q_n \in Q,$ tels que $q_1 = p$, $q_n = r$ et $q_1 \underset{\Delta}{\overset{\varepsilon}{{\to}}} \dots \underset{\Delta}{\overset{\varepsilon}{{\to}}} q_n$.

On notera que $L(\operatorname{cl}_\varepsilon(A))=L(A)$.

\subsection{Projection}

Soit $\Sigma'\subseteq \Sigma$. On d\'efinit la projection $\pi_{\Sigma'}$ de $\Sigma^*$ dans $\Sigma'^*$ comme \'etant l'unique fonction v\'erifiant:

$$\forall u,v \in \Sigma^*, \pi_{\Sigma'}(uv)=\pi_{\Sigma'}(u)\pi_{\Sigma'}(v)$$
$$\pi_{\Sigma'}(\varepsilon)=\varepsilon$$
$$\forall a\in \Sigma,\pi_{\Sigma'}(a)=\left\{\begin{array}{ll}
a &\text{si } a\in \Sigma'\\
\varepsilon &\text{sinon} \\
\end{array}\right.$$

\subsection{Automate de modèle de fautes}
Un automate de modèle de faute est un automate qui reconnait un langage L extension-clos appelé langage de faute. Les états finaux de cet automate sont appelés <<\'etats fautifs>> et les autres <<\'etats normaux>>.
    
    \paragraph{Exemple (automate de mod\`ele de faute): }
    
   Dans la figure \ref{modele de fautes} $f\in \Sigma$ et $L  = \Sigma^*\{f\}\Sigma^*$. Les  nœuds blancs sont les états normaux et les nœuds gris sont les \'etats fautifs. Avec ce mod\`ele de faute, une ex\'ecution est fautive si est seulement si $f$ s'est produit.
        \begin{figure}[H]
                \centering
                \begin{tikzpicture}
                        \node[] (0) at (-1,0) {};
                        \node[draw,circle] (1) at (0,0) {1};
                        \node[draw,circle,fill=gray](2) at (3,0){2};
                        \draw[->,>=latex] (0)->(1);
                        \draw[->,>=latex] (1)->(2) node[midway,above]{$f$};
                        \draw [->,>=latex] (1) edge[in=100,out=80,loop] node[above] {$(\Sigma \backslash \{ f \})^*$} (1);
                        \draw [->,>=latex] (2) edge[in=100,out=80,loop] node[above] {$\Sigma^*$} (2);
                \end{tikzpicture}
                \caption{Automate de mod\`ele de faute simple}
                \label{modele de fautes}
        \end{figure}
        
    

\section{Diagnosticabilit\'e}

Soit un automate fini d\'eterministe $A$ appel\'e syst\`eme, un modèle de faute qui reconnait le langage de faute $L$ et une partition $\Sigma=\Sigma_o\sqcup \Sigma_{no}$. On dira qu'un \'ev\`enement est observable (resp. non-observable) s'il est dans $\Sigma_o$ (resp. $\Sigma_{no}$).

On dit que $L$ est diagnosticable dans le système $A$~\cite{SamSRST96}  si et seulement si $$\begin{array}{l}
\exists n \in \mathbb N, \forall w \in L,\\
\forall x \in \Sigma^*, \left(\left|\pi_{\Sigma_o}(x)\right|\ge n \land wx \in \mathcal L(A)\right) \implies\\
\forall v \in \mathcal L(A), \pi_{\Sigma_o}(v)=\pi_{\Sigma_o}(wx) \implies v \in L
\end{array}$$

Intuitivement, il existe une borne $n$ telle que pour toute ex\'ecution du système produisant un mot $w$ contenant une faute, si l'on attend $n$ nouveaux observables ($x$), alors tous les mots $v$ qui produisent la m\^eme observation que $wx$ contiennent une faute, ce qui veut dire que l'on peut toujours affirmer sans se tromper qu'une faute a eu lieu ($n$ observables après qu'elle se soit produite).

\section{Algorithme de construction d'un diagnostiqueur}
    Nous allons présenter un algorithme pour construire un diagnostiqueur en temps exponentiel. ~\cite{SamSRST96}

    \paragraph{}
    Pour construire un diagnostiqueur de l'automate $A$ avec l'automate de mod\`ele de faute $F$, on proc\'ede de la mani\`ere suivante :
    
\begin{enumerate}
  \item Construire $B=A\times F$, le produit synchrone de $B$ et $F$, qui donne l'automate du langage fautif.
  \item Construire $C=\operatorname{cl}_\varepsilon(B[\varepsilon/no])$ (Nous d\'efinission $B[\varepsilon/no]$ comme l'automate dans lequel nous avons remplac\'e toutes les lettres non-observables dans $\Sigma_{no}$ par $\varepsilon$), nous avons que l'automate $C$ reconna\^it la projection du langage fautif sur l'alphabet des \'ev\`enement observables $\Sigma_o$.
  \item Construire $D=\det(C)$ le d\'eterminis\'e de $C$. On note $Q_D \subset \mathcal{P}(C)$ l'ensemble des états de $D$. Soit $q \in Q_D$, comme $q \subset C$ on peut distinguer trois cas :
  \begin{itemize}
    \item Si tous les \'etats dans $q\subseteq Q$ sont non-fautifs, on dira que $q$ est normal.
    \item Si tous les \'etats dans $q\subseteq Q$ sont fautifs, on dira que $q$ est fautif.
    \item Si $q\subseteq Q$ contient des \'etats fautifs et des \'etats non-fautifs, on dira que $q$ est un \'etat ambigu.
  \end{itemize}
\end{enumerate} 

Nous avons vu qu'en ex\'ecutant un mot par l'automate, en fonction de l'\'etat de l'automate $D$ nous pouvons d\'eduire que si ce prefix du mot contient des \'ev\`enements fautifs (m\^eme si dans certains cas nous ne pouvons pas encore conclure, car il y des <<\'etats ambigus>>). 

\section{D\'ecision de la diagnosticabilit\'e en temps polynomial}

L'id\'ee de l'agorithme est qu'un automate est diagnosticable par rapport \`a un mod\`ele de fautes si et seulement si aucun mot infini contenant une faute n'est ambigu.

\begin{enumerate} 
  \item $\mathcal L (A) \cap \mathcal L(L)$ est le langage des ex\'ecutions avec fautes.
  \item $\pi_{\Sigma_o}(\mathcal L (A) \cap \mathcal L(L))$ est le langage des observations d'ex\'ecutions avec faute.
  \item $\mathcal L(A) \setminus \mathcal L(L)$ est le langage des ex\'ecutions sans faute.
  \item $\pi_{\Sigma_o}(\mathcal L(A) \setminus \mathcal L(L))$ est le langage des observations d'ex\'ecutions sans faute.
  \item Donc $ \mathcal L_{amb} = \pi_{\Sigma_o}(\mathcal L(A) \setminus \mathcal L(L)) \cap \pi_{\Sigma_o}(\mathcal L (A) \cap \mathcal L(L))$ est le langage des observations d'ex\'ecutions ambiguës.
\end{enumerate}
Il suffit donc de v\'erifier que $\mathcal L_{amb}$ ne contient que des mots de taille born\'ee, ce qui est équivalent au fait qu'il soit fini, ce qui est \'equivalent \`a l'absence de cycle accessible dans l'automate. Or on sait construire un automate ($(A\times \overline{\det(L)})[\varepsilon/o] \times (A\times L)[\varepsilon/o]$) de taille $|A|^2 \times |L|^2$ reconnaissant ce langage et la d\'etection de cycle accessible peut se faire en temps lin\'eaire. On dispose donc d'un algorithme en $|A|^2 \times |L|^2$ pour d\'ecider si un syst\`eme est diagnosticable par rapport \`a un mod\`ele de fautes.

Note : La longueur du mot le plus long dans ce langage est la dur\'ee maximale pendant laquelle on ne sait pas si le syst\`eme a fait une faute ou non.
\subsection{Un exemple de la d\'ecision de la diagnosticabilit\'e}

Soit $A$ l'automate \`a diagnostiquer (\`a gauche) et $F$ l'automate de mod\`ele de fautes  (\`a droite) repr\'esent\'es ci dessous avec leurs états finaux grisés :

\begin{figure}[H]
	\centering
    	\begin{tikzpicture}
    	    \node[] (x) at (-1,0) {};
            \node[draw,circle,fill=gray] (0) at (0,0) {0};
            \node[draw,circle,fill=gray] (1) at (1.25,1) {1};
            \node[draw,circle,fill=gray] (2) at (2.25,1) {2};
            \node[draw,circle,fill=gray] (3) at (3.25,1) {3};
            \node[draw,circle,fill=gray] (4) at (1.25,-1) {4};
            \node[draw,circle,fill=gray] (5) at (2.25,-1) {5};
            \draw[->,>=latex] (x)->(0);
            \draw[->,>=latex] (0) -> (1) node[midway,above]{$f$};
            \draw[->,>=latex] (1) -> (2) node[midway,above]{$a$};
            \draw[->,>=latex] (2) -> (3) node[midway,above]{$b$};
            \draw [->,>=latex] (3) edge[in=100,out=80,loop] node[above] {$a,b$} (3);
            \draw[->,>=latex] (0) -> (4) node[midway,above]{$a$};
            \draw[->,>=latex] (4) -> (5) node[midway,above]{$b$};
            \draw [->,>=latex] (5) edge[in=100,out=80,loop] node[above] {$a,b$} (5);            
        \end{tikzpicture}
    	\begin{tikzpicture}
    	    \node[] (x) at (-1,0) {};
        	\node[draw,circle] (1) at (0,0) {0};
            \node[draw,circle,fill=gray](2) at (3,0){1};
            \draw[->,>=latex] (x)->(1);
            \draw[->,>=latex] (1)->(2) node[midway,above]{$f$};
            \draw [->,>=latex] (1) edge[in=100,out=80,loop] node[above] {$a,b$} (1);
            \draw [->,>=latex] (2) edge[in=100,out=80,loop] node[above] {$a,b,f$} (2);
        \end{tikzpicture}
\end{figure}

Etape 2 : Nous faisons le produit synchrone sur les deux automates (\`a gauche), puis la projection sur l'alphabet observable(\`a droite) à fin de faciliter les \'etapes suivantes nous avons renomm\'e les \'etats dans l'automate projet\'e. 



\begin{figure}[H]
	\centering
	\begin{tikzpicture}
	    \node[] (x) at (-1,0) {};
		\node[draw,circle] (0) at (0,0) {0,0};
        \node[draw,circle,fill=gray] (1) at (1.25,1) {1,1};
        \node[draw,circle,fill=gray] (2) at (2.75,1) {2,1};
        \node[draw,circle,fill=gray] (3) at (4.25,1) {3,1};
        \node[draw,circle] (4) at (1.25,-1) {4,0};
        \node[draw,circle] (5) at (2.75,-1) {5,0};
        \draw[->,>=latex] (x)->(0);
        \draw[->,>=latex] (0) -> (1) node[midway,above]{$f$};
        \draw[->,>=latex] (1) -> (2) node[midway,above]{$a$};
        \draw[->,>=latex] (2) -> (3) node[midway,above]{$b$};
        \draw [->,>=latex] (3) edge[in=100,out=80,loop] node[above] {$a,b$} (3);
        \draw[->,>=latex] (0) -> (4) node[midway,above]{$a$};
        \draw[->,>=latex] (4) -> (5) node[midway,above]{$b$};
        \draw [->,>=latex] (5) edge[in=100,out=80,loop] node[above] {$a,b$} (5);
	\end{tikzpicture}
	\begin{tikzpicture}
	    \node[] (x) at (-1,0) {};
		\node[draw,circle] (0) at (0,0) {0};
        \node[draw,circle,fill=gray] (1) at (1.25,1) {1};
        \node[draw,circle,fill=gray] (2) at (2.75,1) {2};
        \node[draw,circle] (3) at (1.25,-1) {3};
        \node[draw,circle] (4) at (2.75,-1) {4};
        \draw[->,>=latex] (x)->(0);
        \draw[->,>=latex] (0) -> (1) node[midway,above]{$a$};
        \draw[->,>=latex] (1) -> (2) node[midway,above]{$b$};
        \draw [->,>=latex] (2) edge[in=100,out=80,loop] node[above] {$a,b$} (3);
        \draw[->,>=latex] (0) -> (3) node[midway,above]{$a$};
        \draw[->,>=latex] (3) -> (4) node[midway,above]{$b$};
        \draw [->,>=latex] (4) edge[in=100,out=80,loop] node[above] {$a,b$} (5);
	\end{tikzpicture}		
\end{figure}
		
Etape 3 : Nous construisons l'automate compl\'ementaire de l'automate de mod\`ele de fautes(\`a gauche), et nous faisons le produit avec l'automate \`a diagnostiquer pour construire l'automate qui reconna\^it $\mathcal L(A) \setminus \mathcal L(L)$ qui est le langage des ex\'ecutions sans faute (\`a droite).

\begin{figure}[H]
	\centering
    	\begin{tikzpicture}
    	    \node[] (x) at (-1,0) {};
        	\node[draw,circle,fill=gray] (1) at (0,0) {0};
            \node[draw,circle](2) at (3,0){1};
            \draw[->,>=latex] (x)->(0);
            \draw[->,>=latex] (1)->(2) node[midway,above]{$f$};
            \draw [->,>=latex] (1) edge[in=100,out=80,loop] node[above] {$a,b$} (1);
            \draw [->,>=latex] (2) edge[in=100,out=80,loop] node[above] {$a,b,f$} (2);
        \end{tikzpicture}
        \begin{tikzpicture}
            \node[] (x) at (-1,0) {};
        	\node[draw,circle,fill=gray] (0) at (0,0) {0,0};
        	\node[draw,circle] (1) at (1.25,1) {1,1};
        	\node[draw,circle] (2) at (2.75,1) {2,1};
        	\node[draw,circle] (3) at (4.25,1) {3,1};
        	\node[draw,circle,fill=gray] (4) at (1.25,-1) {4,0};
        	\node[draw,circle,fill=gray] (5) at (2.75,-1) {5,0};
        	\draw[->,>=latex] (x)->(0);
        	\draw[->,>=latex] (0) -> (1) node[midway,above]{$f$};
        	\draw[->,>=latex] (1) -> (2) node[midway,above]{$a$};
        	\draw[->,>=latex] (2) -> (3) node[midway,above]{$b$};
        	\draw [->,>=latex] (3) edge[in=100,out=80,loop] node[above] {$a,b$} (3);
        	\draw[->,>=latex] (0) -> (4) node[midway,above]{$a$};
        	\draw[->,>=latex] (4) -> (5) node[midway,above]{$b$};
        	\draw [->,>=latex] (5) edge[in=100,out=80,loop] node[above] {$a,b$} (5);
        \end{tikzpicture}
\end{figure}

Etape 4 : Nous renommons les \'etats de l'automate qui reconna\^it $\mathcal L(A) \setminus \mathcal L(L)$(\`a gauche) et nous faisons le produit synchrone de cet automate avec l'automate de l'\'etape 2 qui reconna\^it le langage $\pi_{\Sigma_o}(\mathcal L (A) \cap \mathcal L(L))$.


\begin{figure}[H]
	\centering
        \begin{tikzpicture}
            \node[] (x) at (-1,0) {};
        	\node[draw,circle,fill=gray] (0) at (0,0) {0};
        	\node[draw,circle] (1) at (1.25,1) {1};
        	\node[draw,circle] (2) at (2.75,1) {2};
        	\node[draw,circle] (3) at (4.25,1) {3};
        	\node[draw,circle,fill=gray] (4) at (1.25,-1) {4};
        	\node[draw,circle,fill=gray] (5) at (2.75,-1) {5};
        	\draw[->,>=latex] (x)->(0);
        	\draw[->,>=latex] (0) -> (1) node[midway,above]{$f$};
        	\draw[->,>=latex] (1) -> (2) node[midway,above]{$a$};
        	\draw[->,>=latex] (2) -> (3) node[midway,above]{$b$};
        	\draw [->,>=latex] (3) edge[in=100,out=80,loop] node[above] {$a,b$} (3);
        	\draw[->,>=latex] (0) -> (4) node[midway,above]{$a$};
        	\draw[->,>=latex] (4) -> (5) node[midway,above]{$b$};
        	\draw [->,>=latex] (5) edge[in=100,out=80,loop] node[above] {$a,b$} (5);
        \end{tikzpicture}
        \begin{tikzpicture}
            \node[] (x) at (-1,0) {};
	        \node[draw,circle] (0) at (0,0) {0,0};
        	\node[draw,circle,fill=gray] (1) at (1.25,1) {4,1};
        	\node[draw,circle,fill=gray] (2) at (2.75,1) {5,2};
        	\node[draw,circle] (3) at (1.25,-1) {4,3};
        	\node[draw,circle] (4) at (2.75,-1) {5,4};
        	\draw[->,>=latex] (x)->(0);
        	\draw[->,>=latex] (0) -> (1) node[midway,above]{$a$};
        	\draw[->,>=latex] (1) -> (2) node[midway,above]{$b$};
        	\draw [->,>=latex] (2) edge[in=100,out=80,loop] node[above] {$a,b$} (3);
        	\draw[->,>=latex] (0) -> (3) node[midway,above]{$a$};
        	\draw[->,>=latex] (3) -> (4) node[midway,above]{$b$};
        	\draw [->,>=latex] (4) edge[in=100,out=80,loop] node[above] {$a,b$} (5);
	\end{tikzpicture}
	
	Dans ce dernier automate on peut voir qu'il y a un cycle : $(5,2) \xrightarrow{a,b} (5,2) \xrightarrow{a,b} \dots $.
	
	Donc l'automate $A$ que l'on voulait diagnostiquer n'est pas diagnosticable.
\end{figure}




\section{D\'efinitions de l'opacit\'e}

\subsection{Quelques d\'efinitions de base}

Nous allons d'abord donner quelques d\'efinitions sur le syst\`eme de transition d'\'etats \'etiquet\'e afin d'introduire les conceptions de opacit\'e.

\subsubsection{Syst\`eme de transition d'\'etats \'etiquett\'e (LTS)}

Un syst\`eme de transition d'\'etats \'etiquett\'e LTS est un quadruplet 

$$LTS = (S,L,\Delta,S_0)$$

avec
\begin{itemize}
	\item $S$ : un ensemble d'\'etats (potentiellement infini)
	\item $L$ : un ensemble d'étiquettes(potentiellement infini)
	\item $\Delta \subseteq S\times L \times S$ : une relation de transition
	\item $S_0$ : un ensemble fini non vide des \'etats initiaux 
\end{itemize}

Remarque : dans la suite nous ne consid\'erons que des syst\`eme de transition d'\'etats \'etiquet\'es d\'eterministes.

\subsubsection{Run}	

Un $Run$ d'un LTS est un couple $(s_0,\lambda)$ avec :

\begin{itemize}
	\item $s_0 \in S_0$
	\item $\lambda = l_1 \dots l_n$ : une s\'equence finie d'\'etiquettes telle qu'il existe $s_1,s_2,\dots ,s_n \in S$ tels que $\forall i \in \{1, \dots n\}, (s_{i-1},l_i,s_i) \in \Delta$ 
\end{itemize}

Notons $s_n$ par $s_0\oplus \lambda$, remarquons que dans la sous-section pr\'ec\'edente nous avons fait l'hypoth\`ese que le $LTS$ est d\'eterministe, donc $s_0\oplus \lambda$ est bien d\'efini.

L'ensemble des tous les $Run$ est not\'e par $Run(LTS)$.

Nous notons le langage g\'en\'er\'e par $LTS$ par 

$$\mathcal{L}(LTS) = \{\lambda | \exists s_0 \in S_0 \mbox{ tel que } (s_0,\lambda) \in Run(LTS)\}$$


\subsection{Observabilit\'e et opacit\'e}

Jeremy Bryans, Maciej Koutny, Laurent Mazar{\'{e}}, Peter Y. A. Ryan ont propos\'e dans leur article ~\cite{BryansKMR08}

Soit $LTS=(S,L,\Delta,S_0)$ est un syst\`eme de transition d'\'etats étiquetté, et $L_{obs}$ est un ensemble d'étiquettes $observables$. Nous notons $L_{obs}^{\varepsilon} = L_{obs} \cup \{\varepsilon\}$

Nous allons maintenant mod\'eliser les diff\'erentes capacit\'es d'observation sur le syst\`eme qui est mod\'elis\'e par un LTS. Nous commen\c cons par d\'efinir les diff\'erents fonctions d'observations

\subsubsection{Diff\'erents fonctions d'observation et leur capacit\'e d'observation}

Nous d\'efinissons toutes les fonctions $\Omega : Run(LTS) \rightarrow L_{obs}^*$ comme les fonctions   
 d'observation.
 
Il y a plusieurs types de capacit\'e d'observation : 


$statique/dynamique/orwellian/m-orwellian(m\geq 1)$ :
		
Il existe une <<fonction d'observation>> sur les mots compos\'es par les \'etiquettes $\Omega_{etiq} : L^* \rightarrow L_{obs}^{\varepsilon}$ telle que 
	
	$$\forall (s,\lambda)= (s,l_1\dots l_n) \in Run(LTS), \Omega(s,\lambda) = \Omega_{etiq}(k_1)\circ\Omega_{etiq}(k_2)\circ\dots\circ\Omega_{etiq}(k_n)$$
	
	Ce sont les diff\'erents choix de $(k_1,\dots, k_n)$ qui d\'efinissent la capacit\'e d'observation.


\begin{itemize}
        \item Statique : $k_i = l_i$ qui modélise un observateur sans mémoire.
	\item Dynamique : $k_i = l_1 \dots l_i$ qui modélise un observateur qui ne connaît que le passé.
	\item Orwellien : $k_i = \lambda$ qui modélise un observateur omniscient avec une mémoire infinie.
	\item m-Orwellien : $k_i = l_{max(1,i-m+1)} \dots l_{min(n,i+m-1)}$ qui modélise un observateur qui connaît le passé proche et le futur proche.

\end{itemize}

	Remarque : les observateurs $statiques$ et $orwelliens$ sont des cas particuliers de $m-orwellien$ : $statique$ est $1-orwellien$ et $orwellien$ est $\infty-orwellien$.

	
\subsubsection{Diff\'erents types d'Opacit\'e}

\begin{mydef}
	Soit $\Omega$ une fonction d'observation et $Secret$ un sous-ensemble de $Run(LTS)$. Nous disons que $Secret$ est opaque par rapport \`a $\Omega$ si et seulement si
	
	$$\forall (s,\lambda) \in Secret, \exists (s',\lambda')\not\in Secret \mbox{ tel que } \Omega(s,\lambda) = \Omega (s',\lambda')$$
	
	ou nous pouvons voir cette d\'efinition en terme d'ensembles :
	
	$$\Omega(Secret)\subseteq \Omega(Run(LTS)-Secret)$$
\end{mydef}

\subsubsection{Exemples des syst\`emes opaques}
		\begin{figure}[H]
                \centering
                \begin{tikzpicture}
                        \node[] (x) at (-1,0) {};
                        \node[draw,circle] (0) at (0,0) {0};
                        \node[draw,circle] (1) at (3,2) {1};
                        \node[draw,circle] (2) at (3,-2) {2};
                        \node[draw,circle] (3) at (6,1) {3};
                        \node[draw,circle] (4) at (6,3) {4};
                        \node[draw,circle] (5) at (6,-2) {5};
                        \draw[->,>=latex] (x)->(0);
                        \draw[->,>=latex,line width = 1mm] (0)->(1) node[midway,above]{$h$};
                        \draw[->,>=latex] (0)->(2) node[midway,above]{$p$};
                        \draw[->,>=latex] (1)->(3) node[midway,above]{$a$};
                        \draw[->,>=latex] (1)->(4) node[midway,above]{$b$};
                        \draw[->,>=latex] (2)->(5) node[midway,above]{$a$};
                \end{tikzpicture}
                \caption{$L = \{h,p,a,b\}$,$L_o = {a,b}$, Le langage que nous cherchons \`a rendre opaque est $L = L^*.h.L^*$}
		\end{figure}		
		
Nous pouvons aussi d\'efinir les diff\'erents niveaux d'opacit\'e comme pour les fonction d'observation. Nous disons que $Secret$ est $init-opaque/final-opaque/total-opaque$ s'il existe un sous-ensemble  $Secret'$ de l'ensemble des \'etats de $LTS$ tel que :

$$\forall (s,\lambda) \in Run(LTS), (s, \lambda) \in Secret \Leftrightarrow \sigma \in Secret'$$


C'est le choix de sigma qui va d\'efinir les diff\'erents niveaux d'opacit\'e :

\begin{itemize}
	\item $initial-opaque$ : $\sigma = s$
	\item $final-opaque$ : $\sigma = s\oplus \lambda$
	\item $total-opaque$ : $\sigma = s s_1 \dots s_n \mbox{ avec } s_i = s\oplus(l_1\dots l_n)$
\end{itemize} 

\section{Non-interf\'erence}

La non-interférence permet de garantir que des actions/informations  confidentielles dans un système ne sont pas observables par des utilisateurs qui n'ont pas un niveau d'accréditation suffisant pour en avoir le droit. 

\begin{mydef}
  Soit $LTS = (S,L,\Delta,S_0)$ un syst\`eme de transition d'\'etats \'etiquett\'e avec $L=L_{high}\sqcup L_{low}$. $LTS$ est non-interf\'erent ssi $\pi_{L_{low}}(L(LTS)) \subseteq L(LTS)$, c'est-\`a-dire si pour chaque run de $LTS$, le run o\`u l'on a enlev\'e tous les \'el\'ements de $L_{high}$ est encore un run.
\end{mydef}

Intuitivement, les étiquettes \emph{low} sont visibles par tout le monde alors que les étiquettes \emph{high} sont priv\'es. Un LTS est dit non-interf\'erent si il n'est pas possible de d\'etecter la pr\'esence d'une étiquette high dans un run en observant uniquement les étiquettes low.~\cite{GorrieriV10}

\begin{myth}[\protect{\cite[page 5]{BryansKMR08}}]
  Toute propri\'et\'e d'opacit\'e initiale reposant sur une fonction d'observation statique peut se r\'eduire \`a un probl\`eme  de non-interf\'erence.
  \label{opacity2noninterf}
\end{myth}

L'id\'ee de la preuve est de construire l'automate de la figure \ref{thfig} o\`u les rectanglent repr\'esentent des sous-automates reconnaissant les langages qu'ils contiennet. En notant $X=a(A\setminus L) + ahL$ le langage reconnu par cet automate, cet automate est non-interf\'erent ssi $aA=a(A\setminus L)+aL=\pi_{L_{low}}(X)\subseteq X$ ssi $A\subseteq A\setminus L$ ssi ce LTS avec $L$ comme secret est opaque.

\begin{figure}[H]
  \centering
  \begin{tikzpicture}
    \node[] (x) at (-1,0) {};
    \node[draw,circle] (0) at (0,0) {$q_0$};
    \node[draw,circle] (1) at (2,0) {$q_1$};
    \node[draw,rectangle] (2) at (0,-2) {$A\setminus L$};
    \node[draw,rectangle] (3) at (2,-2) {$L$};
    \draw[->,>=latex] (x)->(0);
    \draw[->,>=latex] (0)->(1) node[midway,above]{$a$};
    \draw[->,>=latex] (0)->(2) node[midway,left]{$a$};
    \draw[->,>=latex] (1)->(3) node[midway,left]{$h$};
    
  \end{tikzpicture}
  \caption{Id\'ee de la preuve du th\'eor\`eme~\ref{opacity2noninterf}}
  \label{thfig}
\end{figure}

\begin{myth}[\protect{\cite[page 5]{BryansKMR08}}]
Tout probl\`eme de non-interf\'erence peut se r\'eduire \`a un probl\`eme d'opacit\'e reposant sur une fonction d'observation statique.
\end{myth}

\section{Conclusion}

Dans cet article on voulait montrer que l'on pouvait modéliser formellement la sécurité de systèmes complexes. Dans un premier temps, nous avons présenté le problème du diagnostic d'un langage rationnel sur un automate vivant. Tous les systèmes ne sont pas diagnosticables mais la diagnosticabilité est décidable, on présente un algorithme polynomial. Dans un second temps, nous avons présenté le problème de l'opacité  pour un observateur d'un langage dans un système. Dans la réalité, les observateurs peuvent se souvenir de toutes leurs observations pour percer le secret dans certains contextes tandis que leur mémoire est très limitée dans d'autres. Pour rendre compte de ce phénomène nous avons présenté plusieurs types d'observateurs plus ou moins puissants. On présente finalement le problème de la non interférence qui est équivalent à celui de l'opacité. Il existe néanmoins des problèmes concrets qui ne se réduisent pas a priori aux modèles que nous avons présenté : Comment modéliser la réparation d'une panne dans le diagnostic ? Comment obtenir des garanties probabilistes sur la diagnosticabilité et l'opacité ?

\bibliography{bibliography}
\end{document}


